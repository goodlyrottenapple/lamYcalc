\newenvironment{bprooftree}
  {\leavevmode\hbox\bgroup}
  {\DisplayProof\egroup}

\section{\texorpdfstring{\(\lambda\)-Y calculus -
Definitions}{\textbackslash{}lambda-Y calculus - Definitions}}\label{lambda-y-calculus---definitions}

\textbf{Syntax (nominal)}

Types:
\[\sigma ::= a\ |\ \sigma \to \sigma \text{ where }a \in \mathcal{TV}\]

Terms:
\[M::= x\ |\ MM\ |\ \lambda x.M\ |\ Y_\sigma \text{ where }x \in Var\]

\textbf{Well typed terms (nominal)}

\begin{center}
    \AxiomC{}
    \LeftLabel{$(var)$}
    \RightLabel{$(x : \sigma \in \Gamma)$}
    \UnaryInfC{$\Gamma \vdash x : \sigma$}
    \DisplayProof
    \hskip 1.5em
    \AxiomC{}
    \LeftLabel{$(Y)$}
    \UnaryInfC{$\Gamma \vdash Y_\sigma : (\sigma \to \sigma) \to \sigma$}
    \DisplayProof
    \vskip 1.5em
    \AxiomC{$\Gamma \cup \{x:\sigma\} \vdash M : \tau$}
    \LeftLabel{$(abs)$}
    \RightLabel{$(x\ \sharp\ \Gamma/ x \not\in Subjects(\Gamma))$}
    \UnaryInfC{$\Gamma \vdash \lambda x. M : \sigma \to \tau$}
    \DisplayProof
    \hskip 1.5em
    \AxiomC{$\Gamma \vdash M : \sigma \to \tau$}
    \AxiomC{$\Gamma \vdash N : \sigma$}
    \LeftLabel{$(app)$}
    \BinaryInfC{$\Gamma \vdash MN : \tau$}
    \DisplayProof
\end{center}

\textbf{\(\beta Y\)-Reduction(nominal, typed)}

\begin{center}
    \AxiomC{$\Gamma \vdash M \Rightarrow M' : \sigma \to \tau$}
    \AxiomC{$\Gamma \vdash N : \sigma$}
    \LeftLabel{$(red_L)$}
    \BinaryInfC{$\Gamma \vdash MN \Rightarrow M'N : \tau$}
    \DisplayProof
    \hskip 1.5em
    \AxiomC{$\Gamma \vdash M : \sigma \to \tau$}
    \AxiomC{$\Gamma \vdash N \Rightarrow N' : \sigma$}
    \LeftLabel{$(red_R)$}
    \BinaryInfC{$\Gamma \vdash MN \Rightarrow M'N : \tau$}
    \DisplayProof
    \vskip 1.5em
    \AxiomC{$\Gamma \cup \{x:\sigma\} \vdash M \Rightarrow M' : \tau$}
    \LeftLabel{$(abs)$}
    \RightLabel{$(x\ \sharp\ \Gamma)$}
    \UnaryInfC{$\Gamma \vdash \lambda x. M \Rightarrow \lambda x. M' : \sigma \to \tau$}
    \DisplayProof
    \hskip 1.5em
    \AxiomC{$\Gamma \cup \{x:\sigma\} \vdash M : \tau$}
    \AxiomC{$\Gamma \vdash N : \sigma$}
    \LeftLabel{$(\beta)$}
    \RightLabel{$(x\ \sharp\ \Gamma, N)$}
    \BinaryInfC{$\Gamma \vdash (\lambda x. M)N \Rightarrow M[N/x] : \tau$}
    \DisplayProof
    \vskip 1.5em
    \AxiomC{$\Gamma \vdash M : \sigma \to \sigma$}
    \LeftLabel{$(Y)$}
    \UnaryInfC{$\Gamma \vdash Y_\sigma M \Rightarrow M (Y_\sigma M) : \sigma$}
    \DisplayProof
\end{center}

\textbf{\(\beta Y\)-Reduction(nominal, untyped)}

\begin{center}
    \AxiomC{$M \Rightarrow M'$}
    \LeftLabel{$(red_L)$}
    \UnaryInfC{$MN \Rightarrow M'N$}
    \DisplayProof
    \hskip 1.5em
    \AxiomC{$N \Rightarrow N'$}
    \LeftLabel{$(red_R)$}
    \UnaryInfC{$MN \Rightarrow M'N$}
    \DisplayProof
    \vskip 1.5em
    \AxiomC{$M \Rightarrow M'$}
    \LeftLabel{$(abs)$}
    \UnaryInfC{$\lambda x. M \Rightarrow \lambda x. M'$}
    \DisplayProof
    \hskip 1.5em
    \AxiomC{}
    \LeftLabel{$(\beta)$}
    \RightLabel{$(x\ \sharp\ N)$}
    \UnaryInfC{$(\lambda x. M)N \Rightarrow M[N/x]$}
    \DisplayProof
    \hskip 1.5em
    \AxiomC{}
    \LeftLabel{$(Y)$}
    \UnaryInfC{$M \Rightarrow M (Y_\sigma M)$}
    \DisplayProof
\end{center}

\begin{center}\rule{0.5\linewidth}{\linethickness}\end{center}

\textbf{Syntax (locally nameless)}

Types:
\[\sigma ::= a\ |\ \sigma \to \sigma \text{ where }a \in \mathcal{TV}\]

Pre-terms:
\[M::= x\ |\ n\ |\ MM\ |\ \lambda M\ |\ Y_\sigma \text{ where }x \in Var \text{ and } n \in Nat\]

\textbf{Open (locally nameless)}

\(M^N \equiv \{0 \to N\}M\\\)

\begin{math}
\{k \to U\}(x) = x\\
\{k \to U\}(n) = \text{if }k = n \text{ then } U \text{ else } n\\
\{k \to U\}(MN) = (\{k \to U\}M)(\{k \to U\}N)\\
\{k \to U\}(\lambda M) = \lambda (\{k+1 \to U\}M)\\
\{k \to U\}(Y \sigma) = Y \sigma
\end{math}

\textbf{Closed terms (locally nameless, cofinite)}

\begin{center}
    \AxiomC{}
    \LeftLabel{$(fvar)$}
    \UnaryInfC{term$(x)$}
    \DisplayProof
    \hskip 1.5em
    \AxiomC{}
    \LeftLabel{$(Y)$}
    \UnaryInfC{term$(Y_\sigma)$}
    \DisplayProof
    \vskip 1.5em
    \AxiomC{$\forall x \not\in L.\ \text{term}(M^x) $}
    \LeftLabel{$(lam)$}
    \RightLabel{(finite $L$)}
    \UnaryInfC{term$(\lambda M)$}
    \DisplayProof
    \hskip 1.5em
    \AxiomC{term$(M)$}
    \AxiomC{term$(M)$}
    \LeftLabel{$(app)$}
    \BinaryInfC{term$(MN)$}
    \DisplayProof
\end{center}

\textbf{\(\beta Y\)-Reduction(locally nameless, cofinite, untyped)}

\begin{center}
    \AxiomC{$M \Rightarrow M'$}
    \AxiomC{term$(N)$}
    \LeftLabel{$(red_L)$}
    \BinaryInfC{$MN \Rightarrow M'N$}
    \DisplayProof
    \hskip 1.5em
    \AxiomC{term$(M)$}
    \AxiomC{$N \Rightarrow N'$}
    \LeftLabel{$(red_R)$}
  see the full list of pins and alternate functions here  \BinaryInfC{$MN \Rightarrow M'N$}
    \DisplayProof
    \vskip 1.5em
    \AxiomC{$\forall x \not\in L.\ M^x \Rightarrow M'^x$}
    \LeftLabel{$(abs)$}
    \RightLabel{(finite $L$)}
    \UnaryInfC{$\lambda M \Rightarrow \lambda M'$}
    \DisplayProof
    \hskip 1.5em
    \AxiomC{term$(\lambda M)$}
    \AxiomC{term$(N)$}
    \LeftLabel{$(\beta)$}
    \BinaryInfC{$(\lambda M)N \Rightarrow M^N$}
    \DisplayProof
    \hskip 1.5em
    \AxiomC{}
    \LeftLabel{$(Y)$}
    \UnaryInfC{$M \Rightarrow M (Y_\sigma M)$}
    \DisplayProof
\end{center}

\newpage

\section{Bindings}\label{bindings}

When describing the (untyped) \(\lambda\)-calculus on paper, the terms
of the \(\lambda\)-calculus are usually inductively defined in the
following way:

\[t::= x\ |\ tt\ |\ \lambda x.t \text{ where }x \in Var\]

This definition of terms yields an induction/recursion principle, which
can be used to define functions over the \(\lambda\)-terms by structural
recursion and prove properties about the \(\lambda\)-terms using
structural induction (recursion and induction being two sides of the
same coin).\\
Functional languages like Haskell or Isabelle include a way of
inductively defining terms and often provide the inductive/recursive
principles, such as pattern matching on the term constructors in
function definitions or inductive rules based on the definition of
terms.

Whilst the definition above describes terms of the lambda calculus,
there are implicit assumptions one makes about the terms, namely, the
\(x\) in the \(\lambda x.t\) case appears bound in \(t\). This means
that while \(x\) and \(y\) might be distinct terms of the
\(\lambda\)-calculus (i.e. \(x \neq y\)), \(\lambda x.x\) and
\(\lambda y.y\) represent the same term, as \(x\) and \(y\) are bound by
the \(\lambda\). Without the notion of \(\alpha\)-equivalence of terms,
one cannot prove any properties of terms involving bound variables, such
as saying that \(\lambda x.x \equiv \lambda y.y\).\\
The solution to this issue is to quotient the terms of the lambda
calculus by the relation of \(\alpha\)-equivalence. This way,
\(\lambda x.x \equiv_\alpha \lambda y.y\), because both terms are part
of the same equivalence class. In an informal setting, reasoning about
lambda terms often involves substitution of one \(\alpha\)-equivalent
term for another implicitly, to avoid issues with bound names appearing
within another term (often referred to as the Barendregt Variable
Convention). Indeed, even the usual definition of substitution uses this
convention in the lambda case, by implicitly assuming the given lambda
term \(\lambda y. s\) can always be swapped out for an alpha equivalent
term \(\lambda y'. s'\), such that \(y'\) satisfies the side conditions:

\[(\lambda y'. s')[t/x] \equiv \lambda y'.(s'[t/x]) \text{ assuming } y' \not\equiv x\text{ and }y' \not\in FV(t)\]

This definition of substitution is therefore a function on
\(\alpha\)-equivalence classes of terms, rather than on the ``raw''
terms. Because of this, the principles of induction/recursion, obtained
from the definition of the ``raw'' terms are no longer sufficient, as
this definition is obviously not structurally recursive. In order to
reason about the term of the \(\lambda\)-calculus formally, we therefore
need a formalization of the terms which provides induction principles
for \(\alpha\)-equivalent terms.

In general, there are two main approaches taken in a rigorous
formalization of the terms of the lambda calculus, namely the concrete
approaches and the higher-order approaches, both described in some
detail below.

\subsection{Concrete approaches}\label{concrete-approaches}

The concrete or first-order approaches usually encode variables using
names (like strings or natural numbers). Encoding of terms and
capture-avoiding substitution must be encoded explicitly. A survey by
@aydemir08 details three main groups of concrete approaches, found in
formalizations of the \(\lambda\)-calculus in the literature:

\subsubsection{Named}\label{named}

This approach generally defines terms in much the same way as the
informal inductive definition given above. Using a functional language,
such as Haskell or ML, such a definition might look like this:

\begin{verbatim}
datatype trm =
  Var name
| App trm trm
| Lam name trm
\end{verbatim}

Since most reasoning about the lambda terms is up to
\(\alpha\)-equivalence, this notion has to be explicitly stated. There
are several ways of doing this, one of which is using nominal sets
(described in the section on nominal sets/Isabelle?). The nominal
package in Isabelle provides tools to automatically define terms with
binders, with notion of alpha equivalence being handled automatically by
the package. Using nominal sets in Isabelle results in a definition of
terms which looks very similar to the informal presentation of the
lambda calculus:

\begin{verbatim}
nominal_datatype trm =
  Var name
| App trm trm
| Lam x::name l::trm  binds x in l
\end{verbatim}

Most importantly, this definition already includes the notion of alpha
equivalence, wherein
\(\\\)\texttt{Lam\ x\ (Var\ x)\ =\ Lam\ y\ (Var\ y)} immediately
follows. The nominal package also provides freshness lemmas and a
strengthened induction principle with name freshness for terms involving
binders.

\subsubsection{Nameless/de Bruijn}\label{namelessde-bruijn}

Using a named representation of the lambda calculus in a fully formal
setting can be inconvenient when dealing with bound variables. For
example, substitution, as described previously, with its side-condition
of freshness of \(y\) in \(x\) and \(t\) is not structurally recursive,
but rather requires well-founded recursion. To avoid this problem in the
definition of substitution, the terms of the lambda calculus can be
encoded using de Bruijn indices:

\begin{verbatim}
datatype trm =
  Var nat
| App trm trm
| Lam trm
\end{verbatim}

This representation of terms uses indices instead of named variables.
The indices are natural numbers, which encode an occurrence of a
variable in a \(\lambda\)-term. For bound variables, the index indicates
which \(\lambda\) it refers to, by encoding the number of lambdas that
are in the scope between the index and the \(\lambda\)-binder. For
example, the term \(\lambda x.\lambda y. yx\) will be represented as
\(\lambda\ \lambda\ 0\ 1\). Here, 0 stands for \(y\), as there are no
binders in scope between itself and the second \(\lambda\) it
corresponds to, and \(1\) corresponds to \(x\), as there is one
\(\lambda\) binder between itself and the \(\lambda\) it corresponds to.
To encode free variables, one simply choses an index greater than the
number of \(\lambda\)'s currently in scope, for example,
\(\lambda\ 4\).\\
Since there are no named variable, there is only one way to represent
any \(\lambda\)-term, and the notion of \(\alpha\)-equivalence is no
longer relevant.

To see that this representation of \(\lambda\)-terms is isomorphic to
the usual named definition, we can define two function \(f\) and \(g\),
which translate the named representation to de Bruijn notation and vice
versa. More precisely, since we are dealing with \(\alpha\)-equivalence
classes, its is an isomorphism between these that we can formalize.

\begin{math}
f_n^m(x) = x
\end{math}

To make things easier, we only consider a canonical representation of
named terms, where we rename named variables, \(x, y, z,...\) to
\(x_0,x_1,x_2,...\) We then use the first \(n\) variables (where \(n\)
is the number of \(\lambda\)-binders) for bound variables, and the rest
for

In their comparison between named vs.~nameless/de Bruijn representations
of lambda terms, @berghofer06 give further details about the definition
of substitution, which no longer needs the variable convention and can
therefore be defined using primitive structural recursion.\\
The main disadvantage of this approach is the relative unreadability of
both the terms and the formulation of properties about these terms. For
example, the substitution lemma, which in the named setting would be
stated as:

\[\text{If }x \neq y\text{ and }x \not\in FV(L)\text{, then }
M[N/x][L/y] \equiv M[L/y][N[L/y]/x].\]

becomes the following statement in the nameless formalization:

\[\text{For all indices }i, j\text{ with }i \leq j\text{, }M[N/i][L/j] = M[L/j + 1][N[L/j - i]/i]\]

Clearly, the first version of this lemma is much more intuitive.

\subsubsection{Locally Nameless}\label{locally-nameless}

The locally nameless approach to minders is a mix of the two previous
approaches. Whilst a named representation uses variables for both free
and bound variables and the nameless encoding uses de Bruijn indices in
both cases as well, a locally nameless encoding distinguishes between
the two types of variables.\\
Free variables are represented by names, much like in the named version,
and bound variables are encoded using de Bruijn indices. A named term,
such as \(\lambda x. xy\), would be represented as \(\lambda 1y\). The
following definition captures the syntax of the locally nameless terms:

\begin{verbatim}
datatype ptrm =
  Fvar name
  BVar nat
| App trm trm
| Lam trm
\end{verbatim}

Note however, that this definition doesn't quite fit the notion of
\(\lambda\)-terms, since a \texttt{pterm} like \texttt{(BVar\ 1)} does
not represent a \(\lambda\)-term, since bound variables can only appear
in the context of a lambda, such as in \texttt{(Lam\ (BVar\ 1))}.

\subsection{Higher-Order approaches}\label{higher-order-approaches}

Unlike concrete approaches to formalizing the lambda calculus, where the
notion of binding and substitution is defined explicitly in the host
language, higher-order formalizations use the function space of the
implementation language, which handles binding. This way, the
definitions of capture avoiding substitution or notion of
\(\alpha\)-equivalence are offloaded onto the meta-language.

\subsubsection{HOAS}\label{hoas}

HOAS, or higher-order abstract syntax {[}@pfenning88, @harper93{]}, is a
framework for defining logics based on the simply typed lambda calculus.
A form of HOAS, introduced by @harper93, called the Logical Framework
(LF) has been implemented as Twelf by @pfenning99.\\
Using HOAS for encoding the \(\lambda\)-calculus comes down to encoding
binders using the meta-language binders. An example from the @polakow15
paper on embedding linear lambda calculus in Haskell encodes lambda
terms as:

\begin{verbatim}
data Exp a where
  Lam :: Exp a -> Exp b -> Exp (a -> b)
  App :: Exp (a -> b) -> Exp a -> Exp b
\end{verbatim}

This definition avoids the need for explicitly defining substitution,
because it uses Haskell's variables, which already include the necessary
definitions. However, using HOAS only works if the notion of
\(\alpha\)-equivalence and substitution of the meta-language coincide
with these notions in the object-language.

\subsubsection{Weak Higher-Order?}\label{weak-higher-order}

\section{References}\label{references}
