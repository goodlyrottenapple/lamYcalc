\documentclass[a4paper, 12pt, twoside]{style/ociamthesis}
% - Customization --------------------------------------------------------------
% - These settings are changed in metadata.yaml
% - You should not touch anything here

\title{A formalization of the \(\lamy\) calculus}            % the title of the thesis

\author{Samuel Balco}          % your name

\college{GTC}        % your college

\supervisor{Faris Abou-Saleh, Luke Ong and Steven Ramsay}  % your supervisor

\degree{MSc in Computer Science}          % the degree
\degreedate{Trinity 2016}  % the degree date

\logofile{style/logobar}

%input macros (i.e. write your own macros file called mymacros.tex
%and uncomment the next line)
% \include{}

% -----------------------------------------------------------------------------
% -- PACKAGES -----------------------------------------------------------------
% -----------------------------------------------------------------------------
\usepackage{framed}
\usepackage{epstopdf}
\usepackage[usenames,dvipsnames]{xcolor}
% Set figure legends and captions to be smaller sized sans serif font
\usepackage[font={footnotesize,sf}]{caption}
\usepackage{float}

\usepackage[titletoc]{appendix}
\usepackage{pdfpages} % incluce pdf files
\usepackage{wallpaper}


% amsthm stuff --------------------------------------------------
\usepackage{amsthm}
\theoremstyle{plain}
\newtheorem{Theorem}{Theorem}[chapter]
\newtheorem{Lemma}{Lemma}[chapter]
\newtheorem*{Proposition}{Proposition}
\newtheorem*{Corollary}{Corollary}
\theoremstyle{definition}
\newtheorem{Definition}{Definition}[chapter]
\newtheorem{Conjecture}{Conjecture}[chapter]
\newtheorem{Example}{Example}[chapter]
\newtheorem{Postulate}{Postulate}[chapter]
\newtheorem{Problem}{Problem}[chapter]
\theoremstyle{remark}
\newtheorem{Case}{Case}[chapter]
\newtheorem*{Remark}{Remark}
\newtheorem*{Note}{Note}

\makeatletter
\def\thm@space@setup{%
  \thm@preskip=2\parskip \thm@postskip=0pt
}
\makeatother

\makeatletter
\renewenvironment{proof}[1][\proofname]{\par
  \vspace{-\topsep}% remove the space after the theorem
  \pushQED{\qed}%
  \normalfont
  \topsep0pt \partopsep0pt % no space before
  \trivlist
  \item[\hskip\labelsep
        \itshape
    #1\@addpunct{.}]\ignorespaces
}{%
  \popQED\endtrivlist\@endpefalse
  \addvspace{6pt plus 6pt} % some space after
}
\makeatother
% -------------------------------------------------------------


% - Font Stuff starts   ---------------------------------------------------------
% \usepackage{xltxtra}

% % \usepackage{lmodern}
% 
% % \usepackage{amssymb,amsmath}
% \usepackage{ifxetex,ifluatex}
% \usepackage{fixltx2e} % provides \textsubscript

% % use upquote if available, for straight quotes in verbatim environments
\IfFileExists{upquote.sty}{\usepackage{upquote}}{}
% % use microtype if available
\IfFileExists{microtype.sty}{%
\usepackage{microtype}
\UseMicrotypeSet[protrusion]{basicmath} % disable protrusion for tt fonts
}{}
\usepackage{lmodern}
\usepackage{amssymb,amsmath}
\usepackage{ifxetex,ifluatex}
\usepackage{fixltx2e} % provides \textsubscript
\ifnum 0\ifxetex 1\fi\ifluatex 1\fi=0 % if pdftex
  \usepackage[T1]{fontenc}
  \usepackage[utf8]{inputenc}
\else % if luatex or xelatex
  \ifxetex
    \usepackage{mathspec}
  \else
    \usepackage{fontspec}
  \fi
  \defaultfontfeatures{Ligatures=TeX,Scale=MatchLowercase}
    \setmainfont[]{Fira Sans Light}
    \setsansfont[]{Fira Sans}
    \setmonofont[Mapping=tex-ansi,Scale=0.8]{FreeMono}
    \setmathfont(Digits,Latin,Greek)[]{Fira Sans Light}

% - Geometry  ------------------------------------------------------------------
\usepackage[margin=3cm]{geometry}

% - Layout ---------------------------------------------------------------------
\usepackage{fancyhdr}
\usepackage{sectsty}
\usepackage{titlesec}
\titleformat{\chapter}{\bfseries\huge}{\thechapter.}{20pt}{\huge}

% - Links ----------------------------------------------------------------------
\PassOptionsToPackage{usenames,dvipsnames}{xcolor} % color is loaded by hyperref
\ifxetex
\usepackage[pdfusetitle,setpagesize=false, % page size defined by xetex
  unicode=false, % unicode breaks when used with xetex
xetex]{hyperref}
\else
  \usepackage[unicode=true]{hyperref}
\fi

% Make sure url breaks
\usepackage{url}
\makeatletter
\g@addto@macro{\UrlBreaks}{\UrlOrds}
\makeatother
\hypersetup{
            pdftitle={A formalization of the \textbackslash{}lamy calculus},
            pdfauthor={Samuel Balco},
            colorlinks=true,
            linkcolor=cyan,
            citecolor=cyan,
            urlcolor=cyan,
            breaklinks=true}
\urlstyle{same}   % don't use monospace font for urls
\usepackage[all]{hypcap}% improve link placement in floats


% better names when cross-referencing with cleveref
\usepackage[nameinlink]{cleveref}
\makeatother
\crefname{listing}{Figure}{Figures}
\Crefname{listing}{Figure}{Figures}
\crefname{chapter}{Chapter}{Chapters}
\Crefname{chapter}{Chapter}{Chapters}
\crefname{section}{Section}{Sections}
\Crefname{section}{Section}{Sections}
\crefname{subsection}{Section}{Sections}
\Crefname{subsection}{Section}{Sections}
\crefname{subsubsection}{Section}{Sections}
\Crefname{subsubsection}{Section}{Sections}
\crefname{figure}{Figure}{Figures} % changes default behavior to Figure. 1
\Crefname{figure}{Figure}{Figures} % changes default behavior to Figure. 1
\crefname{table}{Table}{Tables}
\Crefname{table}{Table}{Tables}
\crefname{subfigure}{Figure}{Figures}
\Crefname{subfigure}{Figure}{Figures}
\crefname{subsubfigure}{Figure}{Figures}
\Crefname{subsubfigure}{Figure}{Figures}
\crefname{appendix}{Appendix}{Appendices}
\Crefname{appendix}{Appendix}{Appendices}

% - Language -------------------------------------------------------------------

% - Bibliography  --------------------------------------------------------------

% - Listings -------------------------------------------------------------------

% - Graphics  ------------------------------------------------------------------
\usepackage{graphicx,grffile}
\makeatletter
\def\maxwidth{\ifdim\Gin@nat@width>\linewidth\linewidth\else\Gin@nat@width\fi}
\def\maxheight{\ifdim\Gin@nat@height>\textheight\textheight\else\Gin@nat@height\fi}
\makeatother
% Scale images if necessary, so that they will not overflow the page
% margins by default, and it is still possible to overwrite the defaults
% using explicit options in \includegraphics[width, height, ...]{}
\setkeys{Gin}{width=\maxwidth,height=\maxheight,keepaspectratio}

% - Other Options --------------------------------------------------------------




% Make links footnotes instead of hotlinks:
\renewcommand{\href}[2]{#2\footnote{\url{#1}}}


\IfFileExists{parskip.sty}{%
\usepackage{parskip}
}{% else
\setlength{\parindent}{0pt}
\setlength{\parskip}{6pt plus 2pt minus 1pt}
}

\setlength{\emergencystretch}{3em}  % prevent overfull lines
\providecommand{\tightlist}{%
  \setlength{\itemsep}{0pt}\setlength{\parskip}{0pt}}

\setcounter{secnumdepth}{5}

% Redefines (sub)paragraphs to behave more like sections
\ifx\paragraph\undefined\else
\let\oldparagraph\paragraph
\renewcommand{\paragraph}[1]{\oldparagraph{#1}\mbox{}}
\fi
\ifx\subparagraph\undefined\else
\let\oldsubparagraph\subparagraph
\renewcommand{\subparagraph}[1]{\oldsubparagraph{#1}\mbox{}}
\fi


% - Add to Header --------------------------------------------------------------
\usepackage{bussproofs}
\usepackage{amsthm}
\usepackage{minted}
\usepackage{tikz}
\usetikzlibrary{arrows,automata, positioning}
\newcommand{\lamy}{\lambda\text{-}Y}
\newcommand{\concat}{\ensuremath{+\!\!\!\!+\,}}
\newcommand{\wf}{\textsf{Wf-ICtxt}\ }
\newcommand{\tocap}{\leadsto\kern-.5ex\cap}
\newcommand{\conR}{\concat_{\kern-1ex R}}
\newcommand{\conL}{\concat_{\kern-1ex L}}
\newcommand{\poplm}{\textsc{PoplMark}}
\renewcommand{\max}{\textsf{max}\ }
\newcommand{\dip}{\textsf{dp}}
\newcommand{\trm}{\textsf{term}}
\newcommand{\fv}{\textsf{FV}}
\let\oldemptyset\emptyset
\let\emptyset\varnothing

\pagenumbering{roman}

\begin{document}

% - Title ----------------------------------------------------------------------
\maketitle

% - Dedication -----------------------------------------------------------------
\begin{dedication}
This is a dedication
\end{dedication}

% - Acknowledgements -----------------------------------------------------------
\begin{acknowledgements}
Say thanks to whoever listened to your rants for 2 months
\end{acknowledgements}


% - Originality ----------------------------------------------------------------
\begin{originality}
This is the statement of originality
\end{originality}


% - Abstract -------------------------------------------------------------------
\begin{abstract}
This is the abstract. For this and the other front-matter options you
can either include the text directly on the metadata file or you can use
in order to include your text.
\end{abstract}



% - Table of Contents  ---------------------------------------------------------
% \begin{romanpages}
 \hypersetup{linkcolor=black} 
\setcounter{page}{1}
\setcounter{tocdepth}{2}
\tableofcontents 
% \end{romanpages}
% - List of Tables -------------------------------------------------------------
% \end{romanpages}
\newpage

 \hypersetup{linkcolor=cyan} 

% - BODY -----------------------------------------------------------------------
\pagenumbering{arabic}
\chapter{Introduction}\label{introduction}

\section{Motivation}\label{motivation}

Formal verification of software is essential in a lot of safety critical
systems in the industry and has been a field of active research in
computer science. One of the main approaches to verification is model
checking, wherein a system specification is checked against certain
correctness properties, by generating a model of the system, encoding
the desired correctness property as a logical formula and then
exhaustively checking whether the given formula is satisfiable in the
model of the system. Big advances in model checking of
1\textsuperscript{st} order (imperative) programs have been made, with
techniques like abstraction refinement and SAT/SMT-solver use, allowing
scalability.\\
Since aspects of functional programming, such as anonymous/\(\lambda\)
functions have gained prominence in mainstream languages such as C++ or
JavaScript and functional languages like Scala, F\# or Haskell have
garnered wider interest, interest in verifying higher-order functional
programs has also grown. Current approaches to formal verification of
such programs usually involve the use of (automatic) theorem provers,
which usually require a lot of user interaction and as a result have not
managed to scale as well as model checking in the 1\textsuperscript{st}
order setting. Using type systems is another way to ensure program
safety, but using expressive-enough types often requires explicit type
annotations, as is the case for dependent-type systems. Simpler type
systems where type inference is decidable can instead prove too coarse,
i.e.~the required properties are difficult to capture in such type
systems. In recent years, advances in higher order model checking (HOMC)
have been made (C.-H. L. Ong (\protect\hyperlink{ref-ong06}{2006}),
Kobayashi (\protect\hyperlink{ref-kobayashi13}{2013}), Ramsay,
Neatherway, and Ong (\protect\hyperlink{ref-ramsay14}{2014}), Tsukada
and Ong (\protect\hyperlink{ref-tsukada14}{2014})), but whilst a lot of
theory has been developed for HOMC, there has been little done in
implementing/mechanizing these results in a fully formal setting of a
theorem prover.

\section{Aims}\label{aims}

The aim of this project is to make a start of mechanizing the proofs
underpinning HOMC approaches using type-checking of higher-order
recursion schemes, by formalizing the \(\lamy\) calculus with the
intersection-type system described by ? and formally proving certain key
properties of the system.\\
The first part of this work focuses on the mechanization aspect of the
simply typed \(\lamy\) calculus in a theorem prover, in a fashion
similar to the \(\poplm\) challenge, by exploring different encodings of
binders in a theorem prover and also the use of different theorem
provers. The project focuses on the engineering choices and
formalization overheads which result from translating the informal
definitions into a fully-formal setting of a theorem prover. The project
is split into roughly two main parts, with the first part exploring and
evaluating different formalizations of the simply-typed \(\lamy\)
calculus together with the proof of the Church Rosser Theorem. The
second part focuses on implementing the intersection-type system for the
\(\lamy\) calculus and formalizing the proof of subject invariance for
this type system. The formalization and engineering choices made in the
implementation of the intersection-type system reflect the survey and
analysis of the different possible choices of mechanization, explored in
the first part of the project.

\section{Main Achievements}\label{main-achievements}

\begin{itemize}
\tightlist
\item
  Formalization of the simply typed \(\lamy\) calculus and proofs of
  confluence in Isabelle, using both Nominal sets and locally nameless
  encoding of binders.
\item
  Formalization of the simply typed \(\lamy\) calculus and proofs of
  confluence in Agda, using a locally nameless encoding of binders
\item
  Analysis and comparison of binder encodings
\item
  Comparison of Agda and Isabelle
\item
  Formalization of an intersection-type system for the \(\lamy\)
  calculus and proof of subject invariance for intersection-types
\end{itemize}

\chapter{Background}\label{background}

\hypertarget{binders}{\section{Binders}\label{binders}}

When describing the (untyped) \(\lambda\)-calculus on paper, the terms
of the \(\lambda\)-calculus are usually inductively defined in the
following way:

\[t::= x\ |\ tt\ |\ \lambda x.t \text{ where }x \in Var\]

This definition of terms yields an induction/recursion principle, which
can be used to define functions over the \(\lambda\)-terms by structural
recursion and prove properties about the \(\lambda\)-terms using
structural induction (recursion and induction being two sides of the
same coin).\\
However, whilst the definition above describes valid terms of the
\(\lambda\)-calculus, there are implicit assumptions one makes about the
terms, namely, the \(x\) in the \(\lambda x.t\) case appears bound in
\(t\). This means that while \(x\) and \(y\) might be distinct terms of
the \(\lambda\)-calculus (i.e. \(x \neq y\)), \(\lambda x.x\) and
\(\lambda y.y\) represent the same term, as \(x\) and \(y\) are bound by
the \(\lambda\). Without the notion of \(\alpha\)-equivalence of terms,
one cannot prove any properties of terms involving bound variables, such
as saying that \(\lambda x.x \equiv \lambda y.y\).

In an informal setting, reasoning with \(\alpha\)-equivalence of terms
is often very implicit, however in a formal setting of theorem provers,
having an inductive definition of ``raw'' \(lambda\)-terms, which are
not \(alpha\)-equivalent, yet reasoning about \(\alpha\)-equivalent
\(\lambda\)-terms poses certain challenges.\\
One of the main problems is the fact that the inductive/recursive
definition does not easily lift to \(alpha\)-equivalent terms. Take a
trivial example of a function on raw terms, which checks whether a
variable appears bound in a given \(\lambda\)-term. Clearly, such
function is well formed for ``raw'' terms, but does not work (or even
make sense) for \(\alpha\)-equivalent terms.\\
Conversely, there are informal definitions over \(\alpha\)-equivalent
terms, which are not straight-forward to define over raw terms. Take the
usual definition of substitution, defined over \(\alpha\)-equivalent
terms, which actually relies on this fact in the following case:

\[(\lambda y'. s')[t/x] \equiv \lambda y'.(s'[t/x]) \text{ assuming } y' \not\equiv x\text{ and }y' \not\in FV(t)\]

Here in the \(\lambda\) case, it is assumed that a given
\(\lambda\)-term \(\lambda y. s\) can always be swapped out for an alpha
equivalent term \(\lambda y'. s'\), such that \(y'\) satisfies the side
condition. The assumption that a bound variable can be swapped out for a
``fresh'' one to avoid name clashes is often referred to as the
Barendregt Variable Convention.

The direct approach of defining ``raw'' terms and an additional notion
of \(\alpha\)-equivalence introduces a lot of overhead when defining
functions, as one either has to use the recursive principles for ``raw''
terms and then show that the function lifts to the \(\alpha\)-equivalent
terms or define functions on \(alpha\)-equivalence classes and prove
that it is well-founded, without being able to rely on the structurally
inductive principles that one gets ``for free'' with the ``raw''
terms.\\
Because of this, the usual informal representation of the
\(\lambda\)-calculus is rarely used in a fully formal setting.

To mitigate the overheads of a fully formal definition of the
\(\lambda\)-calculus, we want to have an encoding of the
\(\lambda\)-terms, which includes the notion of \(\alpha\)-equivalence
whilst being inductively defined, giving us the inductive/recursive
principles for \(alpha\)-equivalent terms directly. This can be achieved
in several different ways. In general, there are two main approaches
taken in a rigorous formalization of the terms of the lambda calculus,
namely the concrete approaches and the higher-order approaches, both
described in some detail below.

\subsection{Concrete approaches}\label{concrete-approaches}

The concrete or first-order approaches usually encode variables using
names (like strings or natural numbers). Encoding of terms and
capture-avoiding substitution must be encoded explicitly. A survey by B.
Aydemir et al. (\protect\hyperlink{ref-aydemir08}{2008}) details three
main groups of concrete approaches, found in formalizations of the
\(\lambda\)-calculus in the literature:

\subsubsection{Named}\label{named}

This approach generally defines terms in much the same way as the
informal inductive definition given above. Using a functional language,
such as Haskell or ML, such a definition might look like this:

\begin{minted}[]{isabelle}
datatype trm =
  Var name
| App trm trm
| Lam name trm
\end{minted}

As was mentioned before, defining ``raw'' terms and the notion of
\(\alpha\)-equivalence of ``raw'' terms separately carries a lot of
overhead in a theorem prover and is therefore not favored.

To obtain an inductive definition of \(\lambda\)-terms with a built in
notion of \(\alpha\)-equivalence, one can instead use nominal sets
(\textbf{described in the section on nominal sets/Isabelle?}). The
nominal package in Isabelle provides tools to automatically define terms
with binders, which generate inductive definitions of
\(\alpha\)-equivalent terms. Using nominal sets in Isabelle results in a
definition of terms which looks very similar to the informal
presentation of the lambda calculus:

\begin{minted}[]{isabelle}
nominal_datatype trm =
  Var name
| App trm trm
| Lam x::name l::trm  binds x in l
\end{minted}

Most importantly, this definition allows one to define functions over
\(\alpha\)-equivalent terms using structural induction. The nominal
package also provides freshness lemmas and a strengthened induction
principle with name freshness for terms involving binders.

\subsubsection{Nameless/de Bruijn}\label{namelessde-bruijn}

Using a named representation of the lambda calculus in a fully formal
setting can be inconvenient when dealing with bound variables. For
example, substitution, as described in the introduction, with its
side-condition of freshness of \(y\) in \(x\) and \(t\) is not
structurally recursive on ``raw'' terms, but rather requires
well-founded recursion over \(\alpha\)-equivalence classes of terms. To
avoid this problem in the definition of substitution, the terms of the
lambda calculus can be encoded using de Bruijn indices:

\begin{minted}[]{isabelle}
datatype trm =
  Var nat
| App trm trm
| Lam trm
\end{minted}

This representation of terms uses indices instead of named variables.
The indices are natural numbers, which encode an occurrence of a
variable in a \(\lambda\)-term. For bound variables, the index indicates
which \(\lambda\) it refers to, by encoding the number of
\(\lambda\)-binders that are in the scope between the index and the
\(\lambda\)-binder the variable corresponds to. For example, the term
\(\lambda x.\lambda y. yx\) will be represented as
\(\lambda\ \lambda\ 0\ 1\). Here, 0 stands for \(y\), as there are no
binders in scope between itself and the \(\lambda\) it corresponds to,
and \(1\) corresponds to \(x\), as there is one \(\lambda\)-binder in
scope. To encode free variables, one simply choses an index greater than
the number of \(\lambda\)'s currently in scope, for example,
\(\lambda\ 4\).

To see that this representation of \(\lambda\)-terms is isomorphic to
the usual named definition, we can define two function \(f\) and \(g\),
which translate the named representation to de Bruijn notation and vice
versa. More precisely, since we are dealing with \(\alpha\)-equivalence
classes, its is an isomorphism between these that we can formalize.

To make things easier, we consider a representation of named terms,
where we map named variables, \(x, y, z,...\) to indexed variables
\(x_1,x_2,x_3,...\). Then, the mapping from named terms to de Bruijn
term is given by \(f\), which we define in terms of an auxiliary
function \(e\):

\begin{align*} 
e_k^m(x_n) &= \begin{cases}
k-m(x_n)-1 & x_n \in \text{dom }m\\
k+n & otherwise
\end{cases}\\
e_k^m(uv) &= e_k^m(u)\ e_k^m(v)\\
e_k^m(\lambda x_n.u) &= \lambda\ e_{k+1}^{m \oplus (x_n,k)}(u)
\end{align*}

Then \(f(t) \equiv e_0^\emptyset(t)\)

The function \(e\) takes two additional parameters, \(k\) and \(m\).
\(k\) keeps track of the scope from the root of the term and \(m\) is a
map from bound variables to the levels they were bound at. In the
variable case, if \(x_n\) appears in \(m\), it is a bound variable, and
it's index can be calculated by taking the difference between the
current index and the index \(m(x_k)\), at which the variable was bound.
If \(x_n\) is not in \(m\), then the variable is encoded by adding the
current level \(k\) to \(n\).\\
In the abstraction case, \(x_n\) is added to \(m\) with the current
level \(k\), possibly overshadowing a previous binding of the same
variable at a different level (like in
\(\lambda x_1. (\lambda x_1. x_1)\)) and \(k\) is incremented, going
into the body of the abstraction.

The function \(g\), taking de Bruijn terms to named terms is a little
more tricky. We need to replace indices encoding free variables (those
that have a value greater than or equal to \(k\), where \(k\) is the
number of binders in scope) with named variables, such that for every
index \(n\), we substitute \(x_m\), where \(m = n-k\), without capturing
these free variables.

We need two auxiliary functions to define \(g\):

\begin{align*} 
h_k^b(n) &= \begin{cases}
x_{n-k} & n \geq k\\
x_{k+b-n-1} & otherwise
\end{cases}\\
h_k^b(uv) &= h_k^b(u)\ h_k^b(v)\\
h_k^b(\lambda u) &= \lambda x_{k+b}.\ h_{k+1}^b(u)
\end{align*}

\begin{align*} 
\Diamond_k(n) &= \begin{cases}
n-k & n \geq k\\
0 & otherwise
\end{cases}\\
\Diamond_k(uv) &= \max (\Diamond_k(u),\ \Diamond_k(v))\\
\Diamond_k(\lambda u) &= \Diamond_{k+1}(u)
\end{align*}

The function \(g\) is then defined as
\(g(t) \equiv h_0^{\Diamond_0(t)+1}(t)\). As mentioned above, the
complicated definition has to do with avoiding free variable capture. A
term like \(\lambda (\lambda\ 2)\) intuitively represents a named
\(\lambda\)-term with two bound variables and a free variable \(x_0\)
according to the definition above. If we started giving the bound
variables names in a naive way, starting from \(x_0\), we would end up
with a term \(\lambda x_0.(\lambda x_1.x_0)\), which is obviously not
the term we had in mind, as \(x_0\) is no longer a free variable. To
ensure we start naming the bound variables in such a way as to avoid
this situation, we use \(\Diamond\) to compute the maximal value of any
free variable in the given term, and then start naming bound variables
with an index one higher than the value returned by \(\Diamond\).

As one quickly notices, a term like \(\lambda x.x\) and \(\lambda y.y\)
have a single unique representation as a de Bruijn term \(\lambda\ 0\).
Indeed, since there are no named variables in a de Bruijn term, there is
only one way to represent any \(\lambda\)-term, and the notion of
\(\alpha\)-equivalence is no longer relevant. We thus get around our
problem of having an inductive principle and \(\alpha\)-equivalent
terms, by having a representation of \(\lambda\)-terms where every
\(\alpha\)-equivalence class of \(\lambda\)-terms has a single
representative term in the de Bruijn notation.

In their comparison between named vs.~nameless/de Bruijn representations
of \(\lambda\)-terms, Berghofer and Urban
(\protect\hyperlink{ref-berghofer06}{2006}) give details about the
definition of substitution, which no longer needs the variable
convention and can therefore be defined using primitive structural
recursion.\\
The main disadvantage of using de Bruijn indices is the relative
unreadability of both the terms and the formulation of properties about
these terms. For example, the substitution lemma, which in the named
setting would be stated as:

\[\text{If }x \neq y\text{ and }x \not\in FV(L)\text{, then }
M[N/x][L/y] \equiv M[L/y][N[L/y]/x].\]

becomes the following statement in the nameless formalization:

\[\text{For all indices }i, j\text{ with }i \leq j\text{, }M[N/i][L/j] = M[L/j + 1][N[L/j - i]/i]\]

Clearly, the first version of this lemma is much more intuitive.

\subsubsection{Locally Nameless}\label{locally-nameless}

The locally nameless approach to binders is a mix of the two previous
approaches. Whilst a named representation uses variables for both free
and bound variables and the nameless encoding uses de Bruijn indices in
both cases as well, a locally nameless encoding distinguishes between
the two types of variables.\\
Free variables are represented by names, much like in the named version,
and bound variables are encoded using de Bruijn indices. By using de
Bruijn indices for bound variables, we again obtain an inductive
definition of terms which are already \(alpha\)-equivalent.

While closed terms, like \(\lambda x.x\) and \(\lambda y.y\) are
represented as de Bruijn terms, the term \(\lambda x.xz\) and
\(\lambda x.xz\) are encoded as \(\lambda\ 0z\). The following
definition captures the syntax of the locally nameless terms:

\begin{minted}[]{isabelle}
datatype ptrm =
  Fvar name
  BVar nat
| App trm trm
| Lam trm
\end{minted}

Note however, that this definition doesn't quite fit the notion of
\(\lambda\)-terms, since a \texttt{pterm} like \texttt{(BVar 1)} does
not represent a \(\lambda\)-term, since bound variables can only appear
in the context of a lambda, such as in \texttt{(Lam (BVar 1))}.\\
The advantage of using a locally nameless definition of
\(\lambda\)-terms is a better readability of such terms, compared to
equivalent de Bruijn terms. Another advantage is the fact that
definitions of functions and reasoning about properties of these terms
is much closer to the informal setting.

\subsection{Higher-Order approaches}\label{higher-order-approaches}

Unlike concrete approaches to formalizing the lambda calculus, where the
notion of binding and substitution is defined explicitly in the host
language, higher-order formalizations use the function space of the
implementation language, which handles binding. HOAS, or higher-order
abstract syntax (F. Pfenning and Elliott
\protect\hyperlink{ref-pfenning88}{1988}, Harper, Honsell, and Plotkin
(\protect\hyperlink{ref-harper93}{1993})), is a framework for defining
logics based on the simply typed lambda calculus. A form of HOAS,
introduced by Harper, Honsell, and Plotkin
(\protect\hyperlink{ref-harper93}{1993}), called the Logical Framework
(LF) has been implemented as Twelf by Frank Pfenning and Schürmann
(\protect\hyperlink{ref-pfenning99}{1999}), which has been previously
used to encode the \(\lambda\)-calculus.\\
Using HOAS for encoding the \(\lambda\)-calculus comes down to encoding
binders using the meta-language binders. This way, the definitions of
capture avoiding substitution or notion of \(\alpha\)-equivalence are
offloaded onto the meta-language. As an example, take the following
definition of terms of the \(\lambda\)-calculus in Haskell:

\begin{minted}[]{haskell}
data Term where
  Var :: Int -> Term
  App :: Term -> Term -> Term
  Lam :: (Term -> Term) -> Term
\end{minted}

This definition avoids the need for explicitly defining substitution,
because it encodes a \(\lambda\)-term as a Haskell function
\texttt{(Term -> Term)}, relying on Haskell's internal substitution and
notion of \(\alpha\)-equivalence. As with the de Bruijn and locally
nameless representations, this encoding gives us inductively defined
terms with a built in notion of \(\alpha\)-equivalence.\\
However, using HOAS only works if the notion of \(\alpha\)-equivalence
and substitution of the meta-language coincide with these notions in the
object-language.

\newpage

\section{Simple types}\label{simple-types}

The simple types presented throughout this work (except for
\protect\hyperlink{itypes}{Chapter 6}) are often referred to as simple
types \emph{a la Curry}, where a simply typed \(\lambda\)-term is a
triple \((\Gamma, M, \sigma)\) s.t. \(\Gamma \vdash M : \sigma\), where
\(\Gamma\) is the typing context, \(M\) is a term of the untyped
\(\lambda\)-calculus and \(\sigma\) is a simple type.\\
In the untyped \(\lambda\)-calculus, simple types and \(\lambda\)-terms
are completely separate, brought together only through the typing
relation \(\vdash\) in the case of simple types \emph{a la Curry}. The
definition of \(\lamy\) terms, however, is dependent on the simple types
in the case of the \(Y\) constants, which are indexed by simple types.
When talking about the \(\lamy\) calculus, we tend to conflate the
``untyped'' \(\lamy\) terms, which are simply the terms defined in
\ref{lamy-trms}, with the ``typed'' \(\lamy\) terms, which are simply
typed terms \emph{a la Curry} of the form \(\Gamma \vdash M : \sigma\),
where \(M\) is an ``untyped'' \(\lamy\) term. Thus, results about the
\(\lamy\) calculus in this work are in fact results about the ``typed''
\(\lamy\) calculus.\\
However, the proofs of the Church Rosser theorem, as presented in the
previous section, use the untyped definition of \(\beta\)-reduction.
Whilst it is possible to define a typed version of \(\beta\)-reduction,
as was demonstrated by the typed version of the \((Y)\) reduction rule,
it turned out to be much easier to first prove the Church Rosser theorem
for the so called ``untyped'' \(\lamy\) calculus and the additionally
restrict this result to only well-types \(\lamy\) terms. Thus, the
definition of the Church Rosser Theorem, formulated for the \(\lamy\)
calculus, is the following one:

\begin{Theorem}[Church Rosser]

\[\Gamma \vdash M : \sigma \land M \Rightarrow^* M' \land M \Rightarrow^* M'' \implies \exists M'''.\ \ M' \Rightarrow^* M''' \land M'' \Rightarrow^* M''' \land \Gamma \vdash M''' : \sigma\]

\end{Theorem}

In order to prove this typed version of the Church Rosser Theorem, we
need to prove an additional result of subject reduction for \(\lamy\)
calculus, namely:

\begin{Theorem}[Subject reduction for $\Rightarrow_\beta$]

\[\Gamma \vdash M : \sigma \land M \Rightarrow^* M' \implies \Gamma \vdash M' : \sigma\]

\end{Theorem}

\section{\texorpdfstring{\(\lamy\)
calculus}{\textbackslash{}lamy calculus}}\label{lamy-calculus}

Originally, the field of higher order model checking mainly involved
studying higher order recursion schemes (HORS), but more recently,
exploring the \(\lamy\) calculus, which is an extension of the simply
typed \(\lambda\)-calculus, in the context of HOMC has gained traction
(Clairambault and Murawski
(\protect\hyperlink{ref-clairambault13}{2013})). We therefore present
the \(\lamy\) calculus, along with the proofs of the Church Rosser
theorem and the formalization of intersection types for the \(\lamy\)
calculus, as the basis for formalizing the theory of HOMC.

\subsection{Definitions}\label{definitions}

The first part of this project focuses on formalizing the simply typed
\(\lamy\) calculus and the proof of confluence for this calculus. The
usual/informal definition of the \(\lamy\) terms and the simple types
are given below:

\begin{Definition}[$\lamy$ types and terms]

Let \(Var\) be a countably infinite set of atoms in the definition of
the set of \(\lambda\)-terms \(M\): \label{lamy-trms}

\begin{align*}
\sigma ::=&\ \mathsf{o}\ |\ \sigma \to \sigma \\
M ::=&\ x\ |\ MM\ |\ \lambda x.M\ |\ Y_\sigma \text{ where }x \in Var \\
\end{align*}

\end{Definition}

The \(\lamy\) calculus differs from the simply typed
\(\lambda\)-calculus only in the addition of the \(Y\) constant family,
indexed at every simple type \(\sigma\), where the (simple) type of a
\(Y_A\) constant (indexed with the type \(A\)) is \((A \to A) \to A\).
The usual definition of \(\beta\)-reduction is then augmented with the
\((Y)\) rule (this is the typed version of the rule):

\begin{center}
    \AxiomC{$\Gamma \vdash M : \sigma \to \sigma$}
    \LeftLabel{$(Y)$}
    \UnaryInfC{$\Gamma \vdash Y_\sigma M \Rightarrow M (Y_\sigma M) : \sigma$}
    \DisplayProof
\end{center}

In essence, the \(Y\) rule allows (some) well-typed recursive
definitions over simply typed \(\lambda\)-terms. Take for example the
term \(\lambda x.x\), commonly referred to as the \emph{identity}. The
\emph{identity} term can be given a type \(\sigma \to \sigma\) for any
simple type \(\sigma\). We can therefore perform the following
(well-typed) reduction in the \(\lamy\) calculus:
\[Y_\sigma (\lambda x.x) \Rightarrow (\lambda x.x)(Y_\sigma (\lambda x.x))\]

The typed version of the rule illustrates the restricted version of
recursion clearly, since a recursive ``\(Y\)-reduction'' will only occur
if the term \(M\) in \(Y_\sigma M\) has the matching type
\(\sigma \to \sigma\) (to \(Y_\sigma\)'s type
\((\sigma \to \sigma) \to \sigma\)), as in the example above. Due to the
type restriction on \(M\), recursion using the \(Y\) constant will be
\textbf{weakly normalizing (this is right? right?)}, which cannot be
said of unrestricted recursion in the untyped \(\lambda\)-calculus.

\hypertarget{cr-def}{\subsection{Church-Rosser Theorem}\label{cr-def}}

The Church-Rosser Theorem states that the \(\beta\)-reduction of the
\(\lambda\)-calculus is confluent, that is, the reflexive-transitive
closure of the \(\beta\)-reduction has the \emph{diamond property}, i.e.
\(\dip(\Rightarrow^*)\), where:

\begin{Definition}[$\dip(R)$]

A relation \(R\) has the \emph{diamond property}, i.e. \(\dip(R)\), iff
\[\forall a, b, c.\ aRb \land aRc \implies \exists d.\ bRd \land cRd\]

\end{Definition}

The proof of confluence of \(\Rightarrow_Y\), the \(\beta Y\)-reduction
defined as the standard \(\beta\)-reduction with the addition of the
aforementioned \((Y)\) rule, formalized in this project, follows a
variation of the Tait-Martin-Löf Proof originally described in Takahashi
(\protect\hyperlink{ref-takahashi95}{1995}) (specifically using the
notes by R. Pollack (\protect\hyperlink{ref-pollack95}{1995})). To show
why following this proof over the traditional proof is beneficial, we
first give a high level overview of how the usual proof proceeds.

\subsubsection{Overview}\label{overview}

In the traditional proof of the Church Rosser theorem, we define a new
reduction relation, called the \emph{parallel} \(\beta\)-reduction
(\(\gg\)), which, unlike the ``plain'' \(\beta\)-reduction satisfies the
\emph{diamond property} (note that we are talking about the ``single
step'' \(\beta\)-reduction and not the reflexive transitive closure).
Once we prove the \emph{diamond property} for \(\gg\), the proof of
\(\dip(\gg^*)\) follows easily. The reason why we prove \(\dip(\gg)\) in
the first place is because the reflexive-transitive closure of \(\gg\)
coincides with the reflexive transitive closure of \(\Rightarrow\) and
it is much easier to prove \(\dip(\gg)\) than trying to prove
\(\dip(\Rightarrow^*)\) directly. The usual proof of the \emph{diamond
property} for \(\gg\) involves a double induction on the shape of the
two parallel \(\beta\)-reductions from \(M\) to \(P\) and \(Q\), where
we try to show that the following diagram can always be ``closed'':

\begin{figure}[h]
\begin{center}
\begin{tikzpicture}[->,>=stealth',shorten >=1pt,auto,node distance=2.8cm,semithick]
  \tikzstyle{every state}=[fill=none,draw=none,text=black]

  \node[state] (A)                                     {$M$};
  \node[state] (B) [below left= 1.3cm and 1.3cm of A]  {$P$};
  \node[state] (C) [below right= 1.3cm and 1.3cm of A] {$Q$};
  \node[state] (D) [below= 3cm of A]                   {$M'$};
 
  \path (A) edge [left]          node [pos=0.4] {$\gg$} (B)
            edge                 node [pos=0.59]           {$\gg$} (C)
        (B) edge [left, dashed]  node           {$\gg$} (D)
        (C) edge [right, dashed] node           {$\gg$} (D);
\end{tikzpicture}
\end{center}
\caption{The diamond property of $\gg$, visualized}
\end{figure}

The Takahashi (\protect\hyperlink{ref-takahashi95}{1995}) proof
simplifies this proof by eliminating the need to do simultaneous
induction on the \(M \gg P\) and \(M \gg Q\) reductions. This is done by
introducing another reduction, referred to as the \emph{maximal
parallel} \(\beta\)-reduction (\(\ggg\)). The idea of using \(\ggg\) is
to show that for every term \(M\) there is a reduct term \(M_{max}\)
s.t. \(M \ggg M_{max}\). We can then separate the ``diamond'' diagram
above into two instances of the following triangle:

\begin{figure}
\begin{center}
\begin{tikzpicture}[->,>=stealth',shorten >=1pt,auto,node distance=2.8cm,semithick]
  \tikzstyle{every state}=[fill=none,draw=none,text=black]

  \node[state] (A)                                    {$M$};
  \node[state] (B) [below left= 1.3cm and 1.3cm of A] {$M'$};
  \node[state] (D) [below= 3cm of A]                  {$M_{max}$};
 
  \path (A) edge [left]         node [pos=0.4] {$\gg$}  (B)
            edge                node           {$\ggg$} (D)
        (B) edge [left, dashed] node           {$\gg$}  (D);
\end{tikzpicture}
\end{center}
\caption{The proof of $\dip(\gg)$ is split into two instances of this triangle}
\label{gggTriangle}
\end{figure}

\subsubsection{\texorpdfstring{Parallel
\(\beta Y\)-reduction}{Parallel \textbackslash{}beta Y-reduction}}\label{parallel-beta-y-reduction}

Having described the high-level overview of the classicval proof and the
reason for following the Takahashi
(\protect\hyperlink{ref-takahashi95}{1995}) proof, we now describe some
of the major lemmas in more detail.\\
Firstly, we give the definition of \emph{parallel \(\beta Y\)-reduction}
\(\gg\) formulated for the terms of the \(\lamy\) calculus, which allows
simultaneous reduction of multiple parts of a term:

\begin{Definition}[$\gg$]

\(\ \)

\begin{center}
    \AxiomC{}
    \LeftLabel{$(refl)$}
    \UnaryInfC{$x \gg x$}
    \DisplayProof
    \hskip 1.5em
    \AxiomC{$M \gg M'$}
    \AxiomC{$N \gg N'$}
    \LeftLabel{$(app)$}
    \BinaryInfC{$MN \gg M'N'$}
    \DisplayProof
    \vskip 1.5em
    \AxiomC{$M \gg M'$}
    \LeftLabel{$(abs)$}
    \UnaryInfC{$\lambda x. M \gg \lambda x. M'$}
    \DisplayProof
    \hskip 1.5em
    \AxiomC{$M \gg M'$}
    \AxiomC{$N \gg N'$}    \LeftLabel{$(\beta)$}
    \BinaryInfC{$(\lambda x. M)N \gg M'[N'/x]$}
    \DisplayProof
    \hskip 1.5em
    \AxiomC{$M \gg M'$}
    \LeftLabel{$(Y)$}
    \UnaryInfC{$Y_\sigma M \gg M' (Y_\sigma M')$}
    \DisplayProof
\end{center}

\end{Definition}

\textbf{Ex'le here!!}

Then, we proceed to define the \emph{maximum parallel reduction}
\(\ggg\), which contracts all redexes in a given term with a single
step:

\begin{Definition}[$\ggg$]

\(\ \)

\begin{center}
    \AxiomC{}
    \LeftLabel{$(refl)$}
    \UnaryInfC{$x \ggg x$}
    \DisplayProof
    \hskip 1.5em
    \AxiomC{$M \ggg M'$}
    \AxiomC{$N \ggg N'$}
    \LeftLabel{$(app)$}
    \RightLabel{($M$ is not a $\lambda$ or $Y$)}
    \BinaryInfC{$MN \ggg M'N'$}
    \DisplayProof
    \vskip 1.5em
    \AxiomC{$M \ggg M'$}
    \LeftLabel{$(abs)$}
    \UnaryInfC{$\lambda x. M \ggg \lambda x. M'$}
    \DisplayProof
    \hskip 1.5em
    \AxiomC{$M \ggg M'$}
    \AxiomC{$N \ggg N'$}    \LeftLabel{$(\beta)$}
    \BinaryInfC{$(\lambda x. M)N \ggg M'[N'/x]$}
    \DisplayProof
    \hskip 1.5em
    \AxiomC{$M \ggg M'$}
    \LeftLabel{$(Y)$}
    \UnaryInfC{$Y_\sigma M \ggg M' (Y_\sigma M')$}
    \DisplayProof
\end{center}

\end{Definition}

This relation differs from \(\gg\) only in the \((app)\) rule, which can
only be applied if \(M\) is not a \(\lambda\) or \(Y\) term.
\textbf{Ex'le here!!}

To prove \(\dip(\gg)\), we first show that there always exists a term
\(M_{max}\) for every term \(M\), where \(M \ggg M_{max}\) is the
maximal parallel reduction which contracts all redexes in \(M\):

\begin{Lemma}[$\exists\ggg$]

\label{max-ex} \(\forall M.\ \exists M_{max}.\ M \ggg M_{max}\)

\end{Lemma}

\begin{proof}

By induction on M.

\end{proof}

Finally, we show that any parallel reduction \(M \gg M'\) can be
``closed'' by reducing to the term \(M_{max}\) where all redexes have
been contracted (as seen in Figure \ref{gggTriangle}):

\begin{Lemma}

\label{max-close}
\(\forall M, M', M_{max}.\ M \ggg M_{max} \land M \gg M' \implies M' \gg M_{max}\)

\end{Lemma}

\begin{proof}

Omitted. Can be found on p.~8 of the R. Pollack
(\protect\hyperlink{ref-pollack95}{1995}) notes.

\end{proof}

\begin{Lemma}

\(\dip(\gg)\).

\end{Lemma}

\begin{proof}

We can now prove \(\dip(\gg)\) by simply applying Lemma \ref{max-close}
twice, namely for any term \(M\) there is an \(M_{max}\) s.t.
\(M \ggg M_{max}\) (by \ref{max-ex}) and for any \(M', M''\) where
\(M \gg M'\) and \(M \gg M''\), it follows by two applications of Lemma
\ref{max-close} that \(M' \gg M_{max}\) and \(M'' \gg M_{max}\).

\end{proof}

\newpage

\section{Intersection types}\label{intersection-types}

For the formalization of intersection types, we initially chose a strict
intersection-type system, presented in the Bakel
(\protect\hyperlink{ref-bakel}{2003}) notes. Intersection types, as
classically presented in Barendregt, Dekkers, and Statman
(\protect\hyperlink{ref-barendregt13}{2013}) as \(\lambda_\cap^{BCD}\),
extend simple types by adding a conjunction to the definition of types:

\begin{Definition}[$\lambda_\cap^{BCD}$ types]

\[\mathcal{T} ::= \phi\ |\ \mathcal{T} \leadsto \mathcal{T}\ |\ \mathcal{T} \cap \mathcal{T}\]

\end{Definition}

We restrict ourselves to a version of intersection types often called
\emph{strict intersection types}. \emph{Strict intersection types} are a
restriction on \(\lambda_\cap^{BCD}\) types, where an intersection of
types can only appear on the left side of an ``arrow'' type:

\begin{Definition}[Strict intersection types]

\begin{align*} 
\mathcal{T}_s &::= \phi\ |\ \mathcal{T} \leadsto \mathcal{T}_s \\ 
\mathcal{T} &::= (\mathcal{T}_s \cap\hdots\cap \mathcal{T}_s)
\end{align*}

\end{Definition}

The following conventions for intersection types are adopted throughout
this section; \(\omega\) stands for the empty intersection and we write
\(\bigcap\tau_i\) for the type \(\tau_1 \cap\hdots\cap \tau_i\). We also
define a relation \(\subseteq\) for intersection types:

\begin{Definition}[$\subseteq$]

This relation is the least pre-order on intersection types s.t.:

\begin{align*} 
\forall\ 1 \leq j \leq i.&\ \ \tau_j \subseteq \bigcap \tau_i \\ 
\forall\ 1 \leq j \leq i.\ \ \tau_j \subseteq \tau &\implies \bigcap \tau_i \subseteq \tau \\
\rho \subseteq \psi \land \tau \subseteq \mu &\implies \psi \leadsto \tau \subseteq \rho \leadsto \mu\\
\end{align*}

\end{Definition}

Note that \(\lamy\) terms are typed with the strict types
\(\mathcal{T}_s\) only. The typing system, presented for the strict
intersection types, is an adaptation of the one found in the R. Pollack
(\protect\hyperlink{ref-pollack95}{1995}) notes, with the addition of
the typing rule for the \(Y\) constants.

\begin{Definition}[Intersection-type assignment]

\(\\\)

\begin{center}
  \AxiomC{$\exists (x, \bigcap\tau_i) \in \Gamma.\ \tau \subseteq \bigcap\tau_i$}
  \LeftLabel{$(var)$}
  \UnaryInfC{$\Gamma \Vdash x : \tau$}
  \DisplayProof
  %------------------------------------
  \hskip 1.5em
  \AxiomC{$\Gamma \Vdash M : \bigcap\tau_i \leadsto \tau$}
  \AxiomC{$\forall\ 1 \leq j \leq i.\ \ \Gamma \Vdash N : \tau_j$}
  \LeftLabel{$(app)$}
  \BinaryInfC{$\Gamma \Vdash MN : \tau$}
  \DisplayProof
  %------------------------------------
  \vskip 1.5em
  \AxiomC{$(x, \bigcap\tau_i),\Gamma \Vdash M : \tau$}
  \LeftLabel{$(abs)$}
  \UnaryInfC{$\Gamma \Vdash \lambda x.M : \bigcap\tau_i \leadsto \tau$}
  \DisplayProof
  %------------------------------------
  \vskip 1.5em
  \AxiomC{}
  \LeftLabel{$(Y)$}
  \RightLabel{$(1 \leq j \leq i)$}
  \UnaryInfC{$\Gamma \Vdash Y_\sigma : (\bigcap\tau_i \leadsto \tau_1 \cap\hdots\cap \bigcap\tau_i \leadsto \tau_i) \leadsto \tau_j$}
  \DisplayProof
  \vskip 1.5em
\end{center}

\end{Definition}

This is the initial definition, used as a basis for the mechanization,
discussed in \protect\hyperlink{itypes}{Chapter 6}. Due to different
obstacles in the formalization of the subject invariance proofs, this
definition, along with the definition of intersection types was amended
several times. The reasons for these changes are documented in
\protect\hyperlink{itypes}{Chapter 6}.\\
The definition above also assumes that the context \(\Gamma\) is
\emph{well-formed}:

\begin{Definition}[Well-formed intersection-type context]

Assuming that \(\Gamma\) is a finite list, consisting of pairs of atoms
\(Var\) and intersection types \(\mathcal{T}\), \(\Gamma\) is a
\emph{well-formed} context iff:\(\\\)

\begin{center}
  \AxiomC{}
  \LeftLabel{$(nil)$}
  \UnaryInfC{$\wf [\ ]$}
  \DisplayProof
  %------------------------------------
  \hskip 1.5em
  \AxiomC{$x \not\in \mathsf{dom}\ \Gamma$}
  \AxiomC{$\wf \Gamma$}
  \LeftLabel{$(cons)$}
  \BinaryInfC{$\wf (x,\bigcap\tau_i),\Gamma$}
  \DisplayProof
  \vskip 1.5em
\end{center}

\end{Definition}

\chapter{Methodology}\label{methodology}

\section{Comparison of
formalizations}\label{comparison-of-formalizations}

The idea of formalizing a functional language in multiple theorem
provers and objectively assessing the merits and pitfalls of the
different formalizations is definitely not a new idea. The most well
known attempt to do so on a larger scale is the \(\poplm\) challenge,
proposed in the ``Mechanized Metatheory for the Masses: The \(\poplm\)
Challenge'' paper by B. E. Aydemir et al.
(\protect\hyperlink{ref-aydemir05}{2005}).\\
Whilst this paper prompted several formalizations of the benchmark typed
\(\lambda\)-calculus, proposed by the authors of the challenge, in
multiple theorem provers, such as Coq, Isabelle, Matita or Twelf, there
seems to have been no attempt made at analyzing and comparing the
different formalizations and drawing any conclusions with regards to the
stated aims of the challenge.

Whilst this project does not aim to answer the same question as the
original challenge, namely:

\begin{quote}
``How close are we to a world where every paper on programming languages
is accompanied by an electronic appendix with machine- checked proofs?''
(B. E. Aydemir et al. (\protect\hyperlink{ref-aydemir05}{2005}))
\end{quote}

It draws inspiration from the criteria for the ``benchmark
mechanization'', specified by the challenge.

The comparison proceeded in two stages of elimination, where the first
stage was a comparison of the two chosen mechanizations of binders for
the \(\lamy\) calculus (\protect\hyperlink{comp-isa}{Chapter 4}), namely
nominal set and locally nameless representations of binders.\\
After choosing the optimal mechanization of binders, the
\protect\hyperlink{comp-agda}{next chapter} then goes on to compare this
mechanization in two different theorem provers, Isabelle and Agda.\\
The ``wining'' theorem prover from this round was then used to formalize
intersection-types and prove subject invariance.

\subsection{Evaluation criteria}\label{evaluation-criteria}

The \(\poplm\) challenge stated three main criteria for evaluating the
submitted mechanizations of the benchmark calculus:

\begin{itemize}
\tightlist
\item
  Mechanization/implementation overheads
\item
  Technology transparency
\item
  Cost of entry
\end{itemize}

To this, we add another criterion:

\begin{itemize}
\tightlist
\item
  Proof automation
\end{itemize}

This project focuses mainly on the tree criteria of mechanization
overheads, technology transparency and automation, since the focus of
our comparison is to chose the best mechanization and theorem prover to
use for implementing intersection types for the \(\lamy\) calculus (and
proving subject invariance). These criteria are described in greater
detail below:

\subsubsection{Mechanization/implementation
overheads}\label{mechanizationimplementation-overheads}

This aspect of the mechanization is explored predominantly in the next
chapter, which compares two different approaches to formalizing binders
in the \(\lamy\) calculus. Binders are an aspect of our chosen
formalization, where mechanization overheads are most apparent, as
binders are usually overlooked to a large extent in informal setting.\\
As was discussed previously, the treatment of binders is a well studied
problem with several viable solutions. In this project, we decided to
use nominal sets and locally nameless representation for binders, due to
several reasons.\\
The choice of nominal sets was tied to the implementation language,
namely Isabelle, which has a well developed
\href{http://www.inf.kcl.ac.uk/staff/urbanc/Nominal/}{nominal sets
library}, maintained by Christian Urban. The appeal of using nominal
sets is of course the touted minimal overheads in comparison to the
informal presentation.\\
The choice of locally nameless encoding, as opposed to using pure de
Bruijn indices, was motivated by the claim that locally nameless
encoding largely mitigates the disadvantages of de Bruin indices
especially when it comes to technology transparency (i.e.~theorems about
locally nameless presentation are much closer in formulation to the
informal presentation than theorems formulated for de Bruijn indices).\\
Both of these choices were guided in part by the initial choice of
implementation language, Isabelle, which was chosen mainly due to
previous experience in mechanizing similar proofs.\\
The comparison between nominal and locally nameless versions of the
\(\lamy\) calculus, presented in \protect\hyperlink{comp-isa}{Chapter
4}, tries to highlight the differences in the two approaches in contrast
to the usual informal reasoning.

\subsubsection{Technology transparency}\label{technology-transparency}

This criterion is discussed mainly in
\protect\hyperlink{comp-agda}{Chapter 5}, which deals with the
comparison of Isabelle and Agda. The choice of the two theorem provers,
but especially of Isabelle, was largely subjective. Having had previous
experience with Isabelle, it was natural to use it initially, to lower
the cost of entry. Initially only using Isabelle for both formalizations
of binders also allowed for a more uniform comparison of the
mechanization overheads.\\
The choice of Agda as the second implementation language was motivated
by Agda having a dependent-type system. As a result, the style of proofs
in Agda seems quite different to Isabelle, since the distinction between
proofs and programs is largely erased. Agda was chosen over Coq, which
is also a dependently-typed language, because it is more ``bare-bones''
and thus seemed more accessible to a novice in dependently-typed
languages. \textbf{Agda also has a higher ``cool''-factor than Coq,
being a newer language. (Is a joke here ok?)}

\subsubsection{Proof automation}\label{proof-automation}

Proof automation ties into both the mechanization overheads and
transparency aspects of a formalization, since high degree of automation
can often result in a more natural/transparent looking proof where the
``menial'' reasoning steps are taken care of by the theorem prover, and
the user only sees the higher-level reasoning of informal proofs.\\
Both following chapters discuss the automation features of Isabelle and
Agda and try to draw comparisons by analyzing the same/equivalent lemmas
in different mechanizations and theorem provers, in terms of automation.
Whilst on paper, Isabelle includes a lot more automation, in the form of
several tactics and automated theorem provers, whereas Agda comes with
only very simple proof search tactics, Agda's more sophisticated
type-system takes on and replicates at least some of the automation seen
in Isabelle.

\hypertarget{comp-isa}{\chapter{Nominal vs.~Locally
nameless}\label{comp-isa}}

This chapter looks at the two different mechanizations of the \(\lamy\)
calculus, introduced in the previous chapter, namely an implementation
of the calculus using nominal sets and a locally nameless (LN)
mechanization. Having presented the two approaches to formalizing
binders in \protect\hyperlink{binders}{Chapter 2}, this chapters
explores the consequences of choosing either mechanization, especially
in terms of technology transparency and overheads introduced as a result
of the chosen mechanization.

\section{\texorpdfstring{Capture-avoiding substitution and
\(\beta\)-reduction}{Capture-avoiding substitution and \textbackslash{}beta-reduction}}\label{capture-avoiding-substitution-and-beta-reduction}

We give a brief overview of the basic definitions of well-typed terms
and \(\beta\)-reduction, specific to both mechanizations.
Unsurprisingly, the only real differences in these definitions appear in
terms involving \(\lambda\)-binders.

\subsection{Nominal sets
representation}\label{nominal-sets-representation}

As was shown already, nominal set representation of terms is largely
identical with the informal definitions, which is the main reason why
this representation was chosen. This section will examine the
implementation of \(\lamy\) calculus in Isabelle, using the Nominal
package.\\
We start, by examining the definition of untyped \(\beta\)-reduction,
defined for the \(\lamy\) calculus:

\begin{Definition}[$\beta$-reduction]

\label{beta-red-nom} \(\\\)

\begin{center}
    \AxiomC{$M \Rightarrow M'$}
    \LeftLabel{$(red_L)$}
    \UnaryInfC{$MN \Rightarrow M'N$}
    \DisplayProof
    \hskip 1.5em
    \AxiomC{$N \Rightarrow N'$}
    \LeftLabel{$(red_R)$}
    \UnaryInfC{$MN \Rightarrow M'N$}
    \DisplayProof
    \vskip 1.5em
    \AxiomC{$M \Rightarrow M'$}
    \LeftLabel{$(abs)$}
    \UnaryInfC{$\lambda x. M \Rightarrow \lambda x. M'$}
    \DisplayProof
    \hskip 1.5em
    \AxiomC{}
    \LeftLabel{$(\beta)$}
    \RightLabel{$(x\ \sharp\ N)$}
    \UnaryInfC{$(\lambda x. M)N \Rightarrow M[N/x]$}
    \DisplayProof
    \hskip 1.5em
    \AxiomC{}
    \LeftLabel{$(Y)$}
    \UnaryInfC{$Y_\sigma M \Rightarrow M (Y_\sigma M)$}
    \DisplayProof
    \vskip 1.5em
\end{center}

\end{Definition}

This definition, with the exception of the added \((Y)\) rule is the
standard definition of the untyped \(\beta\)-reduction found in
literature (\textbf{link?}). The \(\sharp\) symbol is used to denote the
\emph{freshness} relation in nominal set theory. The side-condition
\(x\ \sharp\ N\) in the \((\beta)\) rule can thus be read as ``\(x\) is
fresh in \(N\)'', namely, the atom \(x\) does not appear in \(N\). For a
\(\lambda\)-term \(M\), we have \(x\ \sharp\ M\) iff
\(x \not\in \fv(M)\), where we take the usual definition of \fv:

\begin{Definition}

The inductively defined \(\fv\) is the set of \emph{free variables} of a
\(\lambda\)-term \(M\).

\begin{align*} 
\fv(x) &= \{ x \}\\
\fv(MN) &= \fv(M) \cup \fv(N)\\
\fv(\lambda x. M) &= \fv(M) \setminus \{ x \}\\
\fv(Y_\sigma) &= \emptyset
\end{align*}

\end{Definition}

The definition of substitution, used in the \((\beta)\) rule is also
unchanged with regards to the usual definition (except for the addition
of the \(Y\) case, which is trivial):

\begin{Definition}[Capture-avoiding substitution]

\begin{align*} 
x[S/y] &= \begin{cases}
S & \text{if }x \equiv y\\
x & otherwise
\end{cases}\\
(MN)[S/y] &= (M[S/y])(N[S/y])\\
x\ \sharp\ y , S \implies (\lambda x.M)[S/y] &= \lambda x.(M[S/y])\\
(Y_\sigma)[S/y] &= Y_\sigma
\end{align*}

\end{Definition}

\subsubsection{Nominal Isabelle
implementation}\label{nominal-isabelle-implementation}

Whilst on paper, all these definitions are unchanged from the usual
presentation, there are a few caveats when it comes to actually
implementing these definitions in Isabelle, using the Nominal package.
The declaration of the terms and types is handled using the reserved
keywords \textbf{\texttt{atom\_decl}} and
\textbf{\texttt{nominal\_datatype}}, which are special versions of the
\textbf{\texttt{typedecl}} and \textbf{\texttt{datatype}} primitives,
used in the usual Isabelle/HOL session:

\begin{minted}[]{isabelle}
atom_decl name

nominal_datatype type = O | Arr type type ("_ → _")

nominal_datatype trm =
  Var name
| App trm trm
| Lam x::name l::trm  binds x in l ("Lam [_]. _" [100, 100] 100)
| Y type
\end{minted}

The special \textbf{\texttt{binds \_ in \_}} syntax in the \texttt{Lam}
constructor declares \texttt{x} to be bound in the body \texttt{l},
telling Nominal Isabelle that \texttt{Lam} terms should be
\textbf{?equated up to \(\alpha\)-equivalence?}, where a term
\(\lambda x. x\) and \(\lambda y. y\) are considered equal, because both
\(x\) and \(y\) are bound in the two respective terms, and can both be
\(\alpha\)-converted to the same term, for example \(\lambda z .z\). In
fact, proving such a lemma is trivial:

\begin{minted}[]{isabelle}
lemma "Lam [x]. Var x = Lam [y]. Var y" by simp
\end{minted}

The specialized \textbf{\texttt{nominal\_datatype}} declaration also
generates definitions of free variables/freshness and other
simplification rules. (Note: These can be inspected in Isabelle, using
the \textbf{\texttt{print\_theorems}} command.)

Next, we define capture avoiding substitution, using a
\textbf{\texttt{nominal\_function}} declaration:

\begin{minted}[]{isabelle}
nominal_function
  subst :: "trm ⇒ name ⇒ trm ⇒ trm"  ("_ [_ ::= _]" [90, 90, 90] 90)
where
  "(Var x)[y ::= s] = (if x = y then s else (Var x))"
| "(App t1 t2)[y ::= s] = App (t1[y ::= s]) (t2[y ::= s])"
| "atom x ♯ (y, s) ⟹ (Lam [x]. t)[y ::= s] = Lam [x].(t[y ::= s])"
| "(Y t)[y ::= s] = Y t"
\end{minted}

Whilst using \textbf{\texttt{nominal\_datatype}} is automatic and
requires no user input, the declaration of a function in Nominal
Isabelle is less straightforward. Unlike using the usual
``\textbf{\texttt{fun}}'' declaration of a recursive function in
Isabelle, where the theorem prover can automatically prove properties
like termination or pattern exhaustiveness, there are several goals (13
in the case of the \texttt{subst} definition) which the user has to
manually prove for any function using recursive nominal data types, such
as the \(\lamy\) terms. This turned out to be a bit problematic, as the
goals involved proving properties like:

\begin{minted}[]{idris}
⋀x t xa ya sa ta.
  eqvt_at subst_sumC (t, ya, sa) ⟹
  eqvt_at subst_sumC (ta, ya, sa) ⟹
  atom x ♯ (ya, sa) ⟹ atom xa ♯ (ya, sa) ⟹ 
  [[atom x]]lst. t = [[atom xa]]lst. ta ⟹ 
  [[atom x]]lst. subst_sumC (t, ya, sa) = 
    [[atom xa]]lst. subst_sumC (ta, ya, sa)
\end{minted}

\textbf{do i need to explain what this property is? or is it ok for
illustrative purposes?}

Whilst most of the goals were trivial, proving cases involving
\(\lambda\)-terms involved a substantial understanding of the internal
workings of Isabelle and the Nominal package and as a novice to using
Nominal Isabelle, understanding and proving these properties proved
challenging. The proof script for the definition of substitution was
actually \textbf{lifted/copied?} from the sample document, found in the
Nominal package documentation, which had a definition of substitution
for the untyped \(\lambda\)-calculus similar enough to be adaptable for
the \(\lamy\) calculus.\\
Whilst this formalization required only a handful of other recursive
function definitions, most of which could be copied from the sample
document, in a different theory with significantly more function
definitions, proving such goals from scratch would prove a challenge to
a Nominal Isabelle newcomer as well as a significant implementation
overhead.

\subsection{Locally nameless
representation}\label{locally-nameless-representation}

As we have seen, on paper at least, the definitions of terms and
capture-avoiding substitution, using nominal sets, are unchanged from
the usual informal definitions. The situation is somewhat different for
the locally nameless mechanization. Since the LN approach combines the
named and de Bruijn representations, there are two different
constructors for free and bound variables:

\subsubsection{Pre-terms}\label{pre-terms}

\begin{Definition}[LN pre-terms]

\label{pterms}
\[M::= x\ |\ n\ |\ MM\ |\ \lambda M\ |\ Y_\sigma \text{ where }x \in Var \text{ and } n \in Nat\]

\end{Definition}

Similarly to the de Bruijn presentation of binders, the \(\lambda\)-term
no longer includes a bound variable, so a named representation term
\(\lambda x.x\) becomes \(\lambda 0\) in LN. As was mentioned in
\protect\hyperlink{binders}{Chapter 2}, the set of terms, defined in
\ref{pterms}, is a superset of \(\lamy\) terms and includes terms which
are not well formed \(\lamy\) terms. For example, the term \(\lambda 3\)
is not a well-formed term, since the bound variable index is out of
scope. Since we don't want to work with terms that do not correspond to
\(\lamy\) terms, we have to introduce the notion of a \emph{well-formed
term}, which restricts the set of pre-terms to only those that
correspond to \(\lamy\) terms (i.e.~this inductive definition ensures
that there are no ``out of bounds'' indices in a given pre-term):

\begin{Definition}[Well-formed terms]

We assume that \(L\) is a finite set in the following definition. \(\\\)

\begin{center}
    \AxiomC{}
    \LeftLabel{$(fvar)$}
    \UnaryInfC{$\trm (x)$}
    \DisplayProof
    \hskip 1.5em
    \AxiomC{}
    \LeftLabel{$(Y)$}
    \UnaryInfC{$\trm (Y_\sigma)$}
    \DisplayProof
    \vskip 1.5em
    \AxiomC{$x \not\in FV(M)$}
    \AxiomC{$\trm (M^x)$}
    \LeftLabel{$(lam)$}
    \BinaryInfC{$\trm (\lambda M)$}
    \DisplayProof
    \hskip 1.5em
    \AxiomC{$\trm (M)$}
    \AxiomC{$\trm (M)$}
    \LeftLabel{$(app)$}
    \BinaryInfC{$\trm (MN)$}
    \DisplayProof
    \vskip 1.5em
\end{center}

\end{Definition}

Already, we see that this formalization introduces some overheads with
respect to the informal/nominal encoding of the \(\lamy\) calculus.\\
The upside of this definition of \(\lamy\) terms becomes apparent when
we start thinking about \(\alpha\)-equivalence and capture-avoiding
substitution. Since the LN terms use de Bruijn levels for bound
variables, there is only one way to write the term \(\lambda x.x\) or
\(\lambda y.y\) as a LN term, namely \(\lambda 0\). As the
\(\alpha\)-equivalence classes of named \(\lamy\) terms collapse into a
singleton \(\alpha\)-equivalence class in a LN representation, the
notion of \(\alpha\)-equivalence becomes trivial.

As a result of using LN representation of binders, the notion of
substitution is split into two distinct operations. One operations is
the substitution of bound variables, called \emph{opening}. The other
substitution is defined for free variables.

\begin{Definition}[Opening and substitution]

We will usually assume that \(S\) is a well-formed LN term when proving
properties about substitution and opening. The abbreviation
\(M^N \equiv \{0 \to N\}M\) is used throughout this chapter.

\begin{enumerate}
\def\labelenumi{\roman{enumi})}
\tightlist
\item
  Opening:

  \begin{align*}
  \{k \to S\}x &= x\\
  \{k \to S\}n &= \begin{cases}
  S & \text{if }k \equiv n\\
  n & otherwise
  \end{cases}\\
  \{k \to S\}(MN) &= (\{k \to S\}M)(\{k \to S\}N)\\
  \{k \to S\}(\lambda M) &= \lambda (\{k+1 \to S\}M)\\
  \{k \to S\}Y_\sigma &= Y_\sigma
  \end{align*}
\item
  Substitution:
\end{enumerate}

\begin{align*} 
x[S/y] &= \begin{cases}
S & \text{if }x \equiv y\\
x & otherwise
\end{cases}\\
n[S/y] &= n \\
(MN)[S/y] &= (M[S/y])(N[S/y])\\
(\lambda M)[S/y] &= \lambda. (M[S/y])\\
Y_\sigma[S/y] &= Y_\sigma
\end{align*}

\end{Definition}

Having defined the \emph{open} operation, we turn back to the definition
of well formed terms, specifically to the \((lam)\) rule, which has the
precondition \(\trm (M^x)\). Intuitively, for the given term
\(\lambda M\), the term \(M^x\) is obtained by replacing all indices
bound to the outermost \(\lambda\) by \(x\). Then, if \(M^x\) is well
formed, so is \(\lambda M\).\\
For example, taking the term \(\lambda\lambda 0(z\ 1)\), we can
construct the following proof-tree, showing that the term is well-typed:

\begin{center}
    \AxiomC{}
    \LeftLabel{$(fvar)$}
    \UnaryInfC{$\trm (y)$}

    \AxiomC{}
    \LeftLabel{$(fvar)$}
    \UnaryInfC{$\trm (z)$}
    \AxiomC{}
    \LeftLabel{$(fvar)$}
    \UnaryInfC{$\trm (x)$}

    \LeftLabel{$(app)$}
    \BinaryInfC{$\trm (z\ x)$}
    \LeftLabel{$(app)$}
    \BinaryInfC{$\trm ((0(z\ x))^y)$}
    \LeftLabel{$(lam)$}
    \UnaryInfC{$\trm ((\lambda 0(z\ 1))^x)$}
    \LeftLabel{$(lam)$}
    \UnaryInfC{$\trm (\lambda\lambda 0(z\ 1))$}
    \DisplayProof
\end{center}

We assumed that \(x \not\equiv y \not\equiv z\) in the proof tree above
and thus omitted the \(x \not\in \fv \hdots\) branches, as they are not
important for this example.\\
If on the other hand, we try construct a similar tree for a term which
is obviously not well formed, such as \(\lambda \lambda 2(z\ 1)\), we
get a proof tree with a branch which cannot be closed (i.e.
\(\trm (2)\)):

\begin{center}
    \AxiomC{$\trm (2)$}

    \AxiomC{}
    \LeftLabel{$(fvar)$}
    \UnaryInfC{$\trm (z)$}
    \AxiomC{}
    \LeftLabel{$(fvar)$}
    \UnaryInfC{$\trm (x)$}

    \LeftLabel{$(app)$}
    \BinaryInfC{$\trm (z\ x)$}
    \LeftLabel{$(app)$}
    \BinaryInfC{$\trm ((2(z\ x))^y)$}
    \LeftLabel{$(lam)$}
    \UnaryInfC{$\trm ((\lambda 2(z\ 1))^x)$}
    \LeftLabel{$(lam)$}
    \UnaryInfC{$\trm (\lambda\lambda 2(z\ 1))$}
    \DisplayProof
    \vskip 1.5em
\end{center}

\subsubsection{\texorpdfstring{\(\beta\)-reduction for LN
terms}{\textbackslash{}beta-reduction for LN terms}}\label{beta-reduction-for-ln-terms}

Finally, we examine the formulation of \(\beta\)-reduction in the LN
presentation of the \(\lamy\) calculus. Since we only want to perform
\(\beta\)-reduction on valid \(\lamy\) terms, the inductive definition
of \(\beta\)-reduction in the LN mechanization now includes the
precondition that the terms appearing in the reduction are well formed:

\begin{Definition}[$\beta$-reduction (LN)]

\(L\) is a finite set of atoms in the following definition: \(\\\)

\begin{center}
    \AxiomC{$M \Rightarrow M'$}
    \AxiomC{$\trm (N)$}
    \LeftLabel{$(red_L)$}
    \BinaryInfC{$MN \Rightarrow M'N$}
    \DisplayProof
    \hskip 1.5em
    \AxiomC{$\trm (M)$}
    \AxiomC{$N \Rightarrow N'$}
    \LeftLabel{$(red_R)$}
    \BinaryInfC{$MN \Rightarrow M'N$}
    \DisplayProof
    \vskip 1.5em
    \AxiomC{$ x \not\in \fv(M) \cup \fv(M')$}
    \AxiomC{$M^x \Rightarrow M'^x$}
    \LeftLabel{$(abs)$}
    \BinaryInfC{$\lambda M \Rightarrow \lambda M'$}
    \DisplayProof
    \hskip 1.5em
    \AxiomC{$\trm (\lambda M)$}
    \AxiomC{$\trm (N)$}
    \LeftLabel{$(\beta)$}
    \BinaryInfC{$(\lambda M)N \Rightarrow M^N$}
    \DisplayProof
    \hskip 1.5em
    \AxiomC{$\trm (M)$}
    \LeftLabel{$(Y)$}
    \UnaryInfC{$Y_\sigma M \Rightarrow M (Y_\sigma M)$}
    \DisplayProof
    \vskip 1.5em
\end{center}

\end{Definition}

As expected, the \emph{open} operation is now used instead of
substitution in the \((\beta)\) rule.\\
The \((abs)\) rule is also slightly different, also using the
\emph{open} in its precondition. Intuitively, the usual formulation of
the \((abs)\) rule states that in order to prove that \(\lambda x. M\)
reduces to \(\lambda x. M'\), we can simply ``un-bind'' \(x\) in both
\(M\) and \(M'\) and show that \(M\) reduces to \(M'\). Since in the
usual formulation of the \(\lambda\)-calculus, there is no distinction
between free and bound variables, this change (where \(x\) becomes free)
is implicit. In the LN presentation, however, this operation is made
explicit by opening both \(M\) and \(M'\) with some free variable \(x\)
(not appearing in either \(M\) nor \(M'\)), which replaces the bound
variables/indices (bound to the outermost \(\lambda\)) with \(x\). While
this definition is equivalent to Definition \ref{beta-red-nom}, the
induction principle this definition yields may not always be sufficient,
especially in situations where we want to open up a term with a free
variable which is not only fresh in \(M\) and \(M'\), but possibly in a
wider context (\textbf{refer to lem 2.5.1 abs case}). We therefore
followed the approach of B. Aydemir et al.
(\protect\hyperlink{ref-aydemir08}{2008}) and re-defined the \((abs)\)
rule (and other definitions involving picking fresh free variables)
using \emph{cofinite quantification}:

\begin{center}
    \vskip 1.5em
    \AxiomC{$\forall x \not\in L.\ M^x \Rightarrow M'^x$}
    \LeftLabel{$(abs)$}
    \UnaryInfC{$\lambda M \Rightarrow \lambda M'$}
    \DisplayProof
    \vskip 1.5em
\end{center}

\subsubsection{Implementation details}\label{implementation-details}

Unlike using the nominal package, the implementation of all the
definitions and functions listed for the LN representation is very
straightforward. To demonstrate this, we present the definition of the
\(\beta\)-reduction in the LN mechanization:

\begin{minted}[]{isabelle}
inductive beta_Y :: "ptrm ⇒ ptrm ⇒ bool" (infix "⇒β" 300)
where
  red_L[intro]: "⟦ trm N ; M ⇒β M' ⟧ ⟹ App M N ⇒β App M' N"
| red_R[intro]: "⟦ trm M ; N ⇒β N' ⟧ ⟹ App M N ⇒β App M N'"
| abs[intro]: "⟦ finite L ; (⋀x. x ∉ L ⟹ M^(FVar x) ⇒β M'^(FVar x)) ⟧ ⟹ 
    Lam M ⇒β Lam M'"
| beta[intro]: "⟦ trm (Lam M) ; trm N ⟧ ⟹ App (Lam M) N ⇒β M^N"
| Y[intro]: "trm M ⟹ App (Y σ) M ⇒β App M (App (Y σ) M)"
\end{minted}

\section{Untyped Church Rosser
Theorem}\label{untyped-church-rosser-theorem}

Having described the implementations of the two binder representations
along with some basic definitions, such as capture-avoiding
substitution, we come the the main part of the comparison, namely the
proof of the Church Rosser theorem. This section examines specific
instances of some of the major lemmas which are part of the bigger
result. The general outline of the proof has been outlined in
\protect\hyperlink{cr-def}{Chapter 2}.

\subsection{Typed vs.~untyped proofs}\label{typ-utyp}

As mentioned previously, when talking about the terms of the \(\lamy\)
calculus, we generally refer to simply typed terms, such as
\(\Gamma \vdash \lambda x. Y_\sigma : \tau \to (\sigma \to \sigma) \to \sigma\).
However, the definitions of reduction seen so far and the consecutive
proofs using these definitions don't use simply typed \(\lamy\) terms,
operating instead on untyped terms. The simplest reason why this is the
case is one of convenience and simplicity. As is the case in most proofs
of the Church Rosser Theorem, the result is usually proved for untyped
terms of the \(\lambda\)-calculus and then extended to simply typed
terms by simply restricting the terms we want to reason about. The
theorem holds due to subject reduction, which says that if a term \(M\)
can be given a simple type \(\sigma\) and \(\beta\)-reduces to another
term \(M'\), the new term can still be typed with the same type
\(\sigma\). Further details about the proofs of subject reduction for
the simply typed \(\lamy\) calculus can be found in the next section of
this chapter.\\
Another reason, besides convention is convenience, specifically
succinctness of code, or the lack thereof, when including simple types
in the definition of \(\beta\)-reduction and all the subsequent lemmas
and theorems. Indeed, the choice of excluding typing information
wherever possible has also been an engineering choice to a large degree,
as it is not good practice (in general) to keep and pass around
variables/objects where not needed in classical programming. The same
applies to functional programming and theorem proving especially, where
notation can often be bloated and cumbersome. Whilst it is true that the
implementation of the proofs of Church Rosser theorem might be shorter,
if the typing information was included directly in the definition of
\(\beta\)-reduction, the downside would be an increased complexity of
proofs, resulting in potentially less understandable and maintainable
code.

\hypertarget{comp-agda}{\chapter{Isabelle vs.~Agda}\label{comp-agda}}

The formalization of the terms and reduction rules of the \(\lambda\)-Y
calculus presented here is a locally nameless presentation due to B.
Aydemir et al. (\protect\hyperlink{ref-aydemir08}{2008}). The basic
definitions of \(\lambda\)-terms and \(\beta\)-reduction were borrowed
from an implementation of the \(\lambda\)-calculus with the associated
Church Rosser proof in Agda, by Mu
(\protect\hyperlink{ref-shing-cheng}{2011}).

The proofs of confluence/Church Rosser were formalized using the paper
by R. Pollack (\protect\hyperlink{ref-pollack95}{1995}), which describes
a coarser proof of Church Rosser than the one formalized by Mu
(\protect\hyperlink{ref-shing-cheng}{2011}). This proof uses the notion
of a maximal parallel reduction, introduced by Takahashi
(\protect\hyperlink{ref-takahashi95}{1995}) to simplify the inductive
proof of confluence.

One of the most obvious differences between Agda and Isabelle is the
treatment of functions and proofs in both languages. Whilst in Isabelle,
there is always a clear syntactic distinction between programs and
proofs, Agda's richer dependent-type system allows constructing proofs
as programs. This distinction is especially apparent in inductive
proofs, which have a completely distinct syntax in Isabelle. As proofs
are not objects which can be directly manipulated in Isabelle, to modify
the proof goal, user commands such as \texttt{apply rule} or
\texttt{by auto} are used:

\begin{minted}[]{isabelle}
lemma subst_fresh: "x ∉ FV t ⟹ t[x ::= u] = t"
apply (induct t)
by auto
\end{minted}

In the proof above, the command \texttt{apply (induct t)} takes a proof
object with the goal \texttt{x ∉ FV t ⟹ t[x ::= u] = t}, and applies the
induction principle for \texttt{t}, generating 5 new proof obligations:

\begin{minted}[]{idris}
proof (prove)
goal (5 subgoals):
1. ⋀xa. x ∉ FV (FVar xa) ⟹ FVar xa [x ::= u] = FVar xa
2. ⋀xa. x ∉ FV (BVar xa) ⟹ BVar xa [x ::= u] = BVar xa
3. ⋀t1 t2.
    (x ∉ FV t1 ⟹ t1 [x ::= u] = t1) ⟹
    (x ∉ FV t2 ⟹ t2 [x ::= u] = t2) ⟹
    x ∉ FV (App t1 t2) ⟹ App t1 t2 [x ::= u] = App t1 t2
4. ⋀t. (x ∉ FV t ⟹ t [x ::= u] = t) ⟹ x ∉ FV (Lam t) ⟹ 
    Lam t [x ::= u] = Lam t
5. ⋀xa. x ∉ FV (Y xa) ⟹ Y xa [x ::= u] = Y xa
\end{minted}

These can then discharged by the call to \texttt{auto}, which is another
command that invokes the automatic solver, which tries to prove all the
goals in the given context.

In comparison, in an Agda proof the proof objects are available to the
user directly. Instead of using commands modifying the proof state, one
begins with a definition of the lemma:

\begin{minted}[]{agda}
subst-fresh : ∀ x t u -> (x∉FVt : x ∉ (FV t)) -> (t [ x ::= u ]) ≡ t
subst-fresh x t u x∉FVt = ?
\end{minted}

The \texttt{?} acts as a `hole' which the user needs to fill in, to
construct the proof. Using the emacs/atom agda-mode, once can apply a
case split to \texttt{t}, corresponding to the \texttt{apply (induct t)}
call in Isabelle, generating the following definition:

\begin{minted}[]{agda}
subst-fresh : ∀ x t u -> (x∉FVt : x ∉ (FV t)) -> (t [ x ::= u ]) ≡ t
subst-fresh x (bv i) u x∉FVt = {!   0!}
subst-fresh x (fv x₁) u x∉FVt = {!   1!}
subst-fresh x (lam t) u x∉FVt = {!   2!}
subst-fresh x (app t t₁) u x∉FVt = {!   3!}
subst-fresh x (Y t₁) u x∉FVt = {!   4!}
\end{minted}

When the above definition is compiled, Agda generates 5 goals needed to
`fill' each hole:

\begin{minted}[]{agda}
?0  :  (bv i [ x ::= u ]) ≡ bv i
?1  :  (fv x₁ [ x ::= u ]) ≡ fv x₁
?2  :  (lam t [ x ::= u ]) ≡ lam t
?3  :  (app t t₁ [ x ::= u ]) ≡ app t t₁
?4  :  (Y t₁ [ x ::= u ]) ≡ Y t₁
\end{minted}

As one can see, there is a clear correspondence between the 5 generated
goals in Isabelle and the cases of the Agda proof above.

Due to this correspondence, reasoning in both systems is often largely
similar. Whereas in Isabelle, one modifies the proof indirectly by
issuing commands to modify proof goals, in Agda, one generates proofs
directly by writing a program-as-proof, which satisfies the type
constraints given in the definition.

\section{Automation}\label{automation}

As seen previously, Isabelle includes several automatic provers of
varying complexity, including \texttt{simp}, \texttt{auto},
\texttt{blast}, \texttt{metis} and others. These are tactics/programs
which automatically apply rewrite-rules until the goal is discharged. If
the tactic fails to discharge a goal within a set number of steps, it
stops and lets the user direct the proof. The use of tactics in Isabelle
is common to prove trivial goals, which usually follow from simple
rewriting of definitions or case analysis of certain variables.\\
For example, the proof goal

\begin{minted}[]{idris}
⋀xa. x ∉ FV (FVar xa) ⟹ FVar xa [x ::= u] = FVar xa
\end{minted}

will be proved by first unfolding the definition of substitution for
\texttt{FVar}

\begin{minted}[]{idris}
(FVar xa)[x ::= u] = (if xa = x then u else FVar xa)
\end{minted}

and then deriving \texttt{x ≠ xa} from the assumption
\texttt{x ∉ FV (FVar xa)}. Applying these steps explicitly, we get:

\begin{minted}[]{isabelle}
lemma subst_fresh: "x ∉ FV t ⟹ t[x ::= u] = t"
apply (induct t)
apply (subst subst.simps(1))
apply (drule subst[OF FV.simps(1)])
apply (drule subst[OF Set.insert_iff])
apply (drule subst[OF Set.empty_iff])
apply (drule subst[OF HOL.simp_thms(31)])
...
\end{minted}

where the goal now has the following shape:

\begin{minted}[]{idris}
1. ⋀xa. x ≠ xa ⟹ (if xa = x then u else FVar xa) = FVar xa
\end{minted}

From this point, the simplifier rewrites \texttt{xa = x} to
\texttt{False} and \texttt{(if False then u else FVar xa)} to
\texttt{FVar xa} in the goal. The use of tactics and automated tools is
heavily ingrained in Isabelle and it is actually impossible
(i.e.~impossible for me) to not use \texttt{simp} at this point in the
proof, partly because one gets so used to discharging such trivial goals
automatically and partly because it becomes nearly impossible to do the
last two steps explicitly without having a detailed knowledge of the
available commands and tactics in Isabelle (i.e.~I don't).\\
Doing these steps explicitly, quickly becomes cumbersome, as one needs
to constantly look up the names of basic lemmas, such as
\texttt{Set.empty\_iff}, which is a simple rewrite rule
\texttt{(?c ∈ \{\}) = False}.

Unlike Isabelle, Agda does not include nearly as much automation. The
only proof search tool included with Agda is Agsy, which is similar,
albeit often weaker than the \texttt{simp} tactic. It may therefore seem
that Agda will be much more cumbersome to reason in than Isabelle. This,
however, turns out not to be the case in this formalization, in part due
to Agda's type system and the powerful pattern matching as well as
direct access to the proof goals.

\section{Proofs-as-programs}\label{proofs-as-programs}

As was already mentioned, Agda treats proofs as programs, and therefore
provides direct access to proof objects. In Isabelle, the proof goal is
of the form:

\begin{minted}[]{idris}
lemma x: "assm-1 ⟹ ... ⟹ assm-n ⟹ concl"
\end{minted}

using the `apply-style' reasoning in Isabelle can become burdensome, if
one needs to modify or reason with the assumptions, as was seen in the
example above. In the example, the \texttt{drule} tactic, which is used
to apply rules to the premises rather than the conclusion, was applied
repeatedly. Other times, we might have to use structural rules for
exchange or weakening, which are necessary purely for
\texttt{organizational} purposes of the proof.\\
In Agda, such rules are not necessary, since the example above looks
like a functional definition:

\begin{minted}[]{idris}
x assm-1 ... assm-n = ?
\end{minted}

Here, \texttt{assm-1} to \texttt{assm-n} are simply arguments to the
function x, which expects something of type \texttt{concl} in the place
of \texttt{?}. This presentation allows one to use the given assumptions
arbitrarily, perhaps passing them to another function/proof or
discarding them if not needed.\\
This way of reasoning is also supported in Isabelle to some extent via
the use of the Isar proof language, where (the previous snippet of) the
proof of \texttt{subst\_fresh} can be expressed in the following way:

\begin{minted}[]{isabelle}
lemma subst_fresh': 
  assumes "x ∉ FV t"
  shows "t[x ::= u] = t"
using assms proof (induct t)
case (FVar y)
  from FVar.prems have "x ∉ {y}" unfolding FV.simps(1) .
  then have "x ≠ y" unfolding Set.insert_iff Set.empty_iff HOL.simp_thms(31) .
  then show ?case unfolding subst.simps(1) by simp
next
...
qed
\end{minted}

This representation is more natural (and readable) to humans, as the
assumptions have been separated and can be referenced and used in a
clearer manner. For example, in the line

\begin{minted}[]{isabelle}
from FVar.prems have "x ∉ {y}"
\end{minted}

the premise \texttt{FVar.prems} is added to the context of the goal
\texttt{x ∉ \{y\}}:

\begin{minted}[]{idris}
proof (prove)
using this:
  x ∉ FV (FVar y)

goal (1 subgoal):
 1. x ∉ {y}
\end{minted}

The individual reasoning steps described in the previous section have
also been separated out into `mini-lemmas' (the command \texttt{have}
creates an new proof goal which has to be proved and then becomes
available as an assumption in the current context) along the lines of
the intuitive reasoning discussed initially. While this proof is more
human readable, it is also more verbose and potentially harder to
automate, as generating valid Isar style proofs is more difficult, due
to `Isar-style' proofs being obviously more complex than `apply-style'
proofs.

Whilst using the Isar proof language gives us a finer control and better
structuring of proofs, one still references proofs only indirectly.
Looking at the same proof in Agda, we have the following definition for
the case of free variables:

\begin{minted}[]{agda}
subst-fresh' x (fv y) u x∉FVt = {!   0!}
\end{minted}

\noindent\rule{8cm}{0.4pt}

\begin{minted}[]{agda}
?0  :  fv y [ x ::= u ] ≡ fv y
\end{minted}

The proof of this case is slightly different from the Isabelle proof. In
order to understand why, we need to look at the definition of
substitution for free variables in Agda:

\begin{minted}[]{agda}
fv y [ x ::= u ] with x ≟ y
... | yes _ = u
... | no _ = fv y
\end{minted}

This definition corresponds to the Isabelle definition, however, instead
of using an if-then-else conditional, the Agda definition uses the
\texttt{with} abstraction to pattern match on \texttt{x ≟ y}. The
\texttt{\_≟\_} function takes the arguments \texttt{x} and \texttt{y},
which are natural numbers, and decides syntactic equality, returning a
\texttt{yes p} or \texttt{no p}, where \texttt{p} is the proof object
showing their in/equality.\\
Since the definition of substitution does not require the proof object
of the equality of \texttt{x} and \texttt{y}, it is discarded in both
cases. If \texttt{x} and \texttt{y} are equal, \texttt{u} is returned
(case \texttt{... | yes \_ = u}), otherwise \texttt{fv y} is returned.

In order for Agda to be able to unfold the definition of
\texttt{fv y [ x ::= u ]}, it needs the case analysis on \texttt{x ≟ y}:

\begin{minted}[]{agda}
subst-fresh' x (fv y) u x∉FVt with x ≟ y
... | yes p = {!   0!}
... | no ¬p = {!   1!}
\end{minted}

\noindent\rule{8cm}{0.4pt}

\begin{minted}[]{agda}
?0  :  (fv y [ x ::= u ] | yes p) ≡ fv y
?1  :  (fv y [ x ::= u ] | no ¬p) ≡ fv y
\end{minted}

In the second case, when \texttt{x} and \texttt{y} are different, Agda
can automatically fill in the hole with \texttt{refl}. Notice that
unlike in Isabelle, where the definition of substitution had to be
manually unfolded (the command \texttt{unfolding subst.simps(1)}), Agda
performs type reduction automatically and can rewrite the term
\texttt{(fv y [ x ::= u ] | no .¬p)} to \texttt{fv y} when type-checking
the expression. Since all functions in Agda terminate, this operation on
types is safe (not sure this is clear enough\ldots{} im not entirely
sure why\ldots{} found here:
http://people.inf.elte.hu/divip/AgdaTutorial/Functions.Equality\_Proofs.html\#automatic-reduction-of-types).

For the case where \texttt{x} and \texttt{y} are equal, one can
immediately derive a contradiction from the fact that \texttt{x} cannot
be equal to \texttt{y}, since \texttt{x} is not a free variable in
\texttt{fv y}. The type of false propositions is \texttt{⊥} in Agda.
Given \texttt{⊥}, one can derive any proposition. To derive \texttt{⊥},
we first inspect the type of x∉FVt, which is \texttt{x ∉ y ∷ []}.
Further examining the definition of \texttt{∉}, we find that
\texttt{x ∉ xs = ¬ x ∈ xs}, which further unfolds to
\texttt{x ∉ xs = x ∈ xs → ⊥}. Thus to obtain \texttt{⊥}, we simply have
to show that \texttt{x ∈ xs}, or in this specific instance
\texttt{x ∈ y ∷ []}. The definition of \texttt{∈} is itself just sugar
for \texttt{x ∈ xs = Any (\_≈\_ x) xs}, where \texttt{Any P xs} means
that there is an element of the list \texttt{xs} which satisfies
\texttt{P}. In this instance, \texttt{P = (\_≈\_ x)}, thus an inhabitant
of the type \texttt{Any (\_≈\_ x) (y ∷ [])} can be constructed if one
has a proof that at least one element in \texttt{y ∷ []} is equivalent
to \texttt{x}. As it happens, such a proof was given as an argument in
\texttt{yes p}:

\begin{minted}[]{agda}
False : ⊥
False = x∉FVt (here p)
\end{minted}

The finished case looks like this (note that \texttt{⊥-elim} takes
\texttt{⊥} and produces something of arbitrary type):

\begin{minted}[]{agda}
subst-fresh' x (fv y) u x∉FVt with x ≟ y
... | yes p = ⊥-elim False
  where
  False : ⊥
  False = x∉FVt (here p)
... | no ¬p = refl
\end{minted}

We can even tranform the Isabelle proof to closer match the Agda proof:

\begin{minted}[]{isabelle}
case (FVar y)
  show ?case
  proof (cases "x = y")
  case True
    with FVar have False by simp
    thus ?thesis ..
  next
  case False then show ?thesis unfolding subst.simps(1) by simp
  qed
\end{minted}

We can thus see that using Isar style proofs and Agda reasoning ends up
being rather similar in practice.

\section{Pattern matching}\label{pattern-matching}

Another reason why automation in the form of explicit proof search
tactics needn't play such a significant role in Agda, is the more
sophisticated type system of Agda (compared to Isabelle). Since Agda
uses a dependent type system, there are often instances where the type
system imposes certain constraints on the arguments/assumptions in a
definition/proof and partially acts as a proof search tactic, by guiding
the user through simple reasoning steps. Since Agda proofs are programs,
unlike Isabelle `apply-style' proofs, which are really proof scripts,
one cannot intuitively view and step through the intermediate reasoning
steps done by the user to prove a lemma. The way one proves a lemma in
Agda is to start with a lemma with a `hole', which is the proof goal,
and iteratively refine the goal until this proof object is constructed.
The way Agda's pattern matching makes constructing proofs easier can be
demonstrated with the following example.

The following lemma states that the parallel-\(\beta\) maximal reduction
preserves local closure:

\[t >>> t' \implies \text{term }t \land \text{term }t'\]

For simplicity, we will prove a slightly simpler version, namely:
\(t >>> t' \implies \text{term }t\). For comparison, this is a short,
highly automated proof in Isabelle:

\begin{minted}[]{isabelle}
lemma pbeta_max_trm_r : "t >>> t' ⟹ trm t"
apply (induct t t' rule:pbeta_max.induct)
apply (subst trm.simps, simp)+
by (auto simp add: lam trm.Y trm.app)
\end{minted}

In Agda, we start with the following definition:

\begin{minted}[]{agda}
>>>-Term-l : ∀ {t t'} -> t >>> t' -> Term t
>>>-Term-l t>>>t' = {!   0!}
\end{minted}

\noindent\rule{8cm}{0.4pt}

\begin{minted}[]{agda}
?0  :  Term .t
\end{minted}

Construction of this proof follows the Isabelle script, in that the
proof proceeds by induction on \(t >>> t'\), which corresponds to the
command \texttt{apply (induct t t' rule:pbeta\_max.induct)}. As seen
earlier, induction in Agda simply corresponds to a case split. The
agda-mode in Emacs/Atom can perform a case split automatically, if
supplied with the variable which should be used for the case analysis,
in this case \texttt{t>>>t'}. Note that Agda is very liberal with
variable names, allowing almost any ASCII or Unicode characters, and it
is customary to give descriptive names to the variables, usually
denoting their type. In this instance, \texttt{t>>>t'} is a variable of
type \texttt{t >>> t'}. Due to Agda's relative freedom in variable
names, whitespace is important, as \texttt{t>> t'} is very different
from \texttt{t >> t'}.

\begin{minted}[]{agda}
>>>-Term-l : ∀ {t t'} -> t >>> t' -> Term t
>>>-Term-l refl = {!   0!}
>>>-Term-l reflY = {!   1!}
>>>-Term-l (app x t>>>t' t>>>t'') = {!   2!}
>>>-Term-l (abs L x) = {!   3!}
>>>-Term-l (beta L cf t>>>t') = {!   4!}
>>>-Term-l (Y t>>>t') = {!   5!}
\end{minted}

\noindent\rule{8cm}{0.4pt}

\begin{minted}[]{agda}
?0  :  Term (fv .x)
?1  :  Term (Y .σ)
?2  :  Term (app .m .n)
?3  :  Term (lam .m)
?4  :  Term (app (lam .m) .n)
?5  :  Term (app (Y .σ) .m)
\end{minted}

The newly expanded proof now contains 5 `holes', corresponding to the 5
constructors for the \(>>>\) reduction. The first two goals are trivial,
since any free variable or Y is a closed term. Here, one can use the
agda-mode again, applying `Refine', which is like a simple proof search,
in that it will try to advance the proof by supplying an object of the
correct type for the specified `hole'. Applying `Refine' to
\texttt{\{!\ \ \ 0!\}} and \texttt{\{!\ \ \ 1!\}} yields:

\begin{minted}[]{agda}
>>>-Term-l : ∀ {t t'} -> t >>> t' -> Term t
>>>-Term-l refl = var
>>>-Term-l reflY = Y
>>>-Term-l (app x t>>>t' t>>>t'') = {!   0!}
>>>-Term-l (abs L x) = {!   1!}
>>>-Term-l (beta L cf t>>>t') = {!   2!}
>>>-Term-l (Y t>>>t') = {!   3!}
\end{minted}

\noindent\rule{8cm}{0.4pt}

\begin{minted}[]{agda}
?0  :  Term (app .m .n)
?1  :  Term (lam .m)
?2  :  Term (app (lam .m) .n)
?3  :  Term (app (Y .σ) .m)
\end{minted}

Since the constructor for \texttt{var} is
\texttt{var : ∀ {x} -> Term (fv x)}, it is easy to see that the
\texttt{hole} can be closed by supplying \texttt{var} as the proof of
\texttt{Term (fv .x)}.\\
A more interesting case is the \texttt{app} case, where using `Refine'
yields:

\begin{minted}[]{agda}
>>>-Term-l : ∀ {t t'} -> t >>> t' -> Term t
>>>-Term-l refl = var
>>>-Term-l reflY = Y
>>>-Term-l (app x t>>>t' t>>>t'') = app {!   0!} {!   1!}
>>>-Term-l (abs L x) = {!   2!}
>>>-Term-l (beta L cf t>>>t') = {!   3!}
>>>-Term-l (Y t>>>t') = {!   4!}
\end{minted}

\noindent\rule{8cm}{0.4pt}

\begin{minted}[]{agda}
?0  :  Term .m
?1  :  Term .n
?2  :  Term (lam .m)
?3  :  Term (app (lam .m) .n)
?4  :  Term (app (Y .σ) .m)
\end{minted}

Here, the refine tactic supplied the constructor \texttt{app}, as it's
type \texttt{app : ∀ {e₁ e₂} -> Term e₁ -> Term e₂ -> Term (app e₁ e₂)}
fit the `hole' (\texttt{Term (app .m .n)}), generating two new `holes',
with the goal \texttt{Term .m} and \texttt{Term .n}. However, trying
`Refine' again on either of the `holes' yields no result. This is where
one applies the induction hypothesis, by adding
\texttt{>>>-Term-l t>>>t'} to \texttt{\{!\ \ \ 0!\}} and applying
`Refine' again, which closes the `hole' \texttt{\{!\ \ \ 0!\}}. Perhaps
confusingly, \texttt{>>>-Term-l t>>>t'} produces a proof of
\texttt{Term .m}. To see why this is, one has to inspect the type of
\texttt{t>>>t'} in this context. Helpfully, the agda-mode provides just
this function, which infers the type of \texttt{t>>>t'} to be
\texttt{.m >>> .m'}. Similarly, \texttt{t>>>t''} has the type
\texttt{.n >>> .n'}. Renaming \texttt{t>>>t'} and \texttt{t>>>t''} to
\texttt{m>>>m'} and \texttt{n>>>n'} respectively, now makes the
recursive call obvious:

\begin{minted}[]{agda}
>>>-Term-l : ∀ {t t'} -> t >>> t' -> Term t
>>>-Term-l refl = var
>>>-Term-l reflY = Y
>>>-Term-l (app x m>>>m' n>>>n') = app (>>>-Term-l m>>>m') {!   0!}
>>>-Term-l (abs L x) = {!   1!}
>>>-Term-l (beta L cf t>>>t') = {!   2!}
>>>-Term-l (Y t>>>t') = {!   3!}
\end{minted}

\noindent\rule{8cm}{0.4pt}

\begin{minted}[]{agda}
?0  :  Term .n
?1  :  Term (lam .m)
?2  :  Term (app (lam .m) .n)
?3  :  Term (app (Y .σ) .m)
\end{minted}

The goal \texttt{Term .n} follows in exactly the same fashion. Applying
`Refine' to the next `hole' yields:

\begin{minted}[]{agda}
>>>-Term-l : ∀ {t t'} -> t >>> t' -> Term t
>>>-Term-l refl = var
>>>-Term-l reflY = Y
>>>-Term-l (app x m>>>m' n>>>n') = app (>>>-Term-l m>>>m') (>>>-Term-l n>>>n')
>>>-Term-l (abs L x) = lam {!   0!} {!   1!}
>>>-Term-l (beta L cf t>>>t') = {!   2!}
>>>-Term-l (Y t>>>t') = {!   3!}
\end{minted}

\noindent\rule{8cm}{0.4pt}

\begin{minted}[]{agda}
?0  :  FVars
?1  :  {x = x₁ : ℕ} → x₁ ∉ ?0 L x → Term (.m ^' x₁)
?2  :  Term (app (lam .m) .n)
?3  :  Term (app (Y .σ) .m)
\end{minted}

At this stage, the interesting goal is \texttt{?1}, due to the fact that
it is dependent on \texttt{?0}. Indeed, replacing \texttt{?0} with
\texttt{L} (which is the only thing of the type \texttt{FVars} available
in this context) changes goal \texttt{?1} to
\texttt{\{x = x₁ : ℕ\} → x₁ ∉ L → Term (.m \textasciicircum' x₁)}:

\begin{minted}[]{agda}
>>>-Term-l : ∀ {t t'} -> t >>> t' -> Term t
>>>-Term-l refl = var
>>>-Term-l reflY = Y
>>>-Term-l (app x m>>>m' n>>>n') = app (>>>-Term-l m>>>m') (>>>-Term-l n>>>n')
>>>-Term-l (abs L x) = lam L {!   0!}
>>>-Term-l (beta L cf t>>>t') = {!   1!}
>>>-Term-l (Y t>>>t') = {!   2!}
\end{minted}

\noindent\rule{8cm}{0.4pt}

\begin{minted}[]{agda}
?0  :  {x = x₁ : ℕ} → x₁ ∉ L → Term (.m ^' x₁)
?1  :  Term (app (lam .m) .n)
?2  :  Term (app (Y .σ) .m)
\end{minted}

Since the goal/type of \texttt{\{!\ \ \ 0!\}} is
\texttt{\{x = x₁ : ℕ\} → x₁ ∉ L → Term (.m \textasciicircum' x₁)},
applying `Refine' will generate a lambda expression
\texttt{(λ x∉L → \{!\ \ \ 0!\})}, as this is obviously the only
`constructor' for a function type. Again, confusingly, we supply the
recursive call \texttt{>>>-Term-l (x x∉L)} to \texttt{\{!\ \ \ 0!\}}. By
examining the type of \texttt{x}, we get that \texttt{x} has the type
\texttt{\{x = x₁ : ℕ\} → x₁ ∉ L → (.m \textasciicircum' x₁) >>> (.m' \textasciicircum' x₁)}.
Then \texttt{(x x∉L)} is clearly of the type
\texttt{(.m \textasciicircum' x₁) >>> (.m' \textasciicircum' x₁)}. Thus
\texttt{>>>-Term-l (x x∉L)} has the desired type
\texttt{Term (.m \textasciicircum' .x)} (note that \texttt{.x} and
\texttt{x} are not the same in this context).

Doing these steps explicitly was not in fact necessary, as the automatic
proof search `Agsy' is capable of automatically constructing proof
objects for all of the cases above. Using `Agsy' in both of the last two
cases, the completed proof is given below:

\begin{minted}[]{agda}
>>>-Term-l : ∀ {t t'} -> t >>> t' -> Term t
>>>-Term-l refl = var
>>>-Term-l reflY = Y
>>>-Term-l (app x m>>>m' n>>>n') = app (>>>-Term-l m>>>m') (>>>-Term-l n>>>n')
>>>-Term-l (abs L x) = lam L (λ x∉L → >>>-Term-l (x x∉L))
>>>-Term-l (beta L cf t>>>t') = app 
  (lam L (λ {x} x∉L → >>>-Term-l (cf x∉L))) 
  (>>>-Term-l t>>>t')
>>>-Term-l (Y t>>>t') = app Y (>>>-Term-l t>>>t')
\end{minted}

\newpage

\hypertarget{itypes}{\chapter{Intersection types}\label{itypes}}

\begin{Definition}[Intersection Types]

Note that \(\mathsf{o}\) and \(\phi\) are constants. \(\omega\) is used
to denote an empty list of strict intersection types. The following
sugar notation will also occasionally be used:
\(\bigcap \tau \equiv [ \tau ]\) and
\(\tau \cap \tau' \equiv \bigcap \tau \concat \bigcap \tau' \equiv [ \tau, \tau' ]\).

\begin{enumerate}
\def\labelenumi{\roman{enumi})}
\tightlist
\item
  Simple types: \[\sigma ::= \mathsf{o}\ |\ \sigma \to \sigma\]
\item
  Intersection types:
  \[\mathcal{T}_s ::= \phi\ |\ \mathcal{T} \leadsto \mathcal{T}\]
  \[\mathcal{T} ::= \mathsf{List}\ \mathcal{T}_s\]
\end{enumerate}

\end{Definition}

The reason why \(\mathcal{T}\) is defined as a list of strict types
\(\mathcal{T}_s\) is due to the requirement that the types in
\(\mathcal{T}\) be finite. The decision to use lists was taken because
the Agda standard library includes a definition of lists along with
definitions of list membership \(\in\) for lists and other associated
lemmas.

Next, we redefine the \(\lambda\)-terms slightly, by annotating the
terms with simple types. The reason for this will be clear later on.

\begin{Definition}[Terms]

Let \(\sigma\) range over simple types in the following definition:

\begin{enumerate}
\def\labelenumi{\roman{enumi})}
\tightlist
\item
  Simply-typed terms:
  \[M::= x_\sigma\ |\ MM\ |\ \lambda x_\sigma.M\ |\ Y_\sigma \text{ where }x \in Var\]
\item
  Simply-typed pre-terms:
  \[M'::= x_\sigma\ |\ i\ |\ M'M'\ |\ \lambda_\sigma.M'\ |\ Y_\sigma \text{ where }x \in Var\text{ and }i \in \mathbb{N}\]
\end{enumerate}

\end{Definition}

Note that both definitions implicitly assume that in the case of
application, a well formed simply-typed term will be of the form \(st\),
where \(s\) has some simple type \(A \to B\) and \(t\) is typed with the
simple type \(A\). Sometimes the same subscript notation will be used to
indicate the simple type of a given pre-/term, for example:
\(m_{A \to B}\). Also, rather confusingly, the simple type of \(Y_A\) is
\((A \to A) \to A\), and thus \(Y_A\) should not be confused with a
constant \(Y\) having a simple type \(A\). \textbf{Maybe use something
like this instead?:} \(m_{:A \to B}\) i.e. \(Y_{A:(A \to A) \to A}\).\\
The typed versions of substitution and the open and close operations are
virtually identical to the untyped versions.

\section{Type refinement}\label{type-refinement}

Next, we introduce the notion of type refinement by defining the
refinement relation \(::\), between simple types and intersection types.

\begin{Definition}[$::$]

Since intersection types are defined in terms of strict
(\(\mathcal{T}_s\)) and non-strict (\(\mathcal{T}\)) intersection types,
for correct typing, the definition of \(::\) is split into two versions,
one for strict and another for non-strict types. In the definition
below, \(\tau\) ranges over strict intersection types \(\mathcal{T}_s\),
with \(\tau_i, \tau_j\) ranging over non-strict intersection types
\(\mathcal{T}\), and \(A, B\) range over simple types \(\sigma\):

\begin{center}
  \AxiomC{}
  \LeftLabel{$(base)$}
  \UnaryInfC{$\phi ::_s \mathsf{o}$}
  \DisplayProof
  \hskip 1.5em
  \AxiomC{$\tau_i :: A$}
  \AxiomC{$\tau_j :: B$}
  \LeftLabel{$(arr)$}
  \BinaryInfC{$\tau_i \leadsto \tau_j ::_s A \to B$}
  \DisplayProof
  \vskip 1.5em
  \AxiomC{}
  \LeftLabel{$(nil)$}
  \UnaryInfC{$\omega :: A$}
  \DisplayProof
  \hskip 1.5em
  \AxiomC{$\tau ::_s A$}
  \AxiomC{$\tau_i :: A$}
  \LeftLabel{$(cons)$}
  \BinaryInfC{$\tau , \tau_i :: A$}
  \DisplayProof
\end{center}

\end{Definition}

Having a notion of refinement, we define a restricted version of a
subset relation on intersection types, which is defined only for pairs
of intersection types, which refine the same simple type.

\begin{Definition}[$\subseteq^A$]

In the definition below, \(\tau, \tau'\) range over \(\mathcal{T}_s\),
\(\tau_i, \hdots, \tau_n\) range over \(\mathcal{T}\) and \(A, B\) range
over \(\sigma\):

\begin{center}
  \AxiomC{}
  \LeftLabel{$(base)$}
  \UnaryInfC{$\phi \subseteq^\mathsf{o}_s \phi$}
  \DisplayProof
  \hskip 1.5em
  \AxiomC{$\tau_i \subseteq^A \tau_j$}
  \AxiomC{$\tau_m \subseteq^B \tau_n$}
  \LeftLabel{$(arr)$}
  \BinaryInfC{$\tau_j \leadsto \tau_m \subseteq^{A \to B}_s \tau_i \leadsto \tau_n$}
  \DisplayProof
  \vskip 1.5em
  \AxiomC{$\tau_i :: A$}
  \LeftLabel{$(nil)$}
  \UnaryInfC{$\omega \subseteq^A \tau_i$}
  \DisplayProof
  \hskip 1.5em
  \AxiomC{$\exists \tau' \in \tau_j.\ \tau \subseteq^A_s \tau'$}
  \AxiomC{$\tau_i \subseteq^A \tau_j$}
  \LeftLabel{$(cons)$}
  \BinaryInfC{$\tau , \tau_i \subseteq^A \tau_j$}
  \DisplayProof
  \vskip 1.5em
  \AxiomC{$(\tau_i \leadsto (\tau_j \concat \tau_k) ,\ \tau_m) :: A \to B$}
  \LeftLabel{$(\tocap)$}
  \UnaryInfC{$(\tau_i \leadsto (\tau_j \concat \tau_k) ,\ \tau_m) \subseteq^{A \to B} (\tau_i \leadsto \tau_j ,\ \tau_i \leadsto \tau_k ,\ \tau_m)$}
  \DisplayProof
  \hskip 1.5em
  \AxiomC{$\tau_i \subseteq^A \tau_j$}
  \AxiomC{$\tau_j \subseteq^A \tau_k$}
  \LeftLabel{$(trans)$}
  \BinaryInfC{$\tau_i \subseteq^A \tau_k$}
  \DisplayProof
  \vskip 1.5em
\end{center}

\end{Definition}

It's easy to show the following properties hold for the \(\subseteq^A\)
and \(::\) relations:

\begin{Lemma}[$\subseteq\implies::$]

\begin{enumerate}
\def\labelenumi{\roman{enumi})}
\tightlist
\item
  \(\tau \subseteq^A_s \delta \implies \tau ::_s A \land \delta ::_s A\)
\item
  \(\tau_i \subseteq^A \delta_i \implies \tau_i :: A \land \delta_i :: A\)
\end{enumerate}

\end{Lemma}

\begin{proof}

\label{test} By \textbf{?mutual?} induction on the relations
\(\subseteq^A_s\) and \(\subseteq^A\).

\end{proof}

\textbf{Lemma} (\(\subseteq\) admissible) The following rules are
admissible in \(\subseteq^A_s/\subseteq^A\):

\begin{enumerate}
\def\labelenumi{\roman{enumi})}
\item
  \AxiomC{$\tau ::_s A$} \LeftLabel{$(refl_s)$}
  \UnaryInfC{$\tau \subseteq^A_s \tau$} \DisplayProof
   \hskip 1.5em \AxiomC{$\tau_i :: A$} \LeftLabel{$(refl)$}
  \UnaryInfC{$\tau_i \subseteq^A \tau_i$} \DisplayProof
   \hskip 1.5em \AxiomC{$\tau \subseteq^A_s \tau'$}
  \AxiomC{$\tau' \subseteq^A_s \tau''$} \LeftLabel{$(trans_s)$}
  \BinaryInfC{$\tau \subseteq^A_s \tau''$} \DisplayProof
   \hskip 1.5em \AxiomC{$\tau_i \subseteq \tau_j$}
  \LeftLabel{$(\subseteq)$} \RightLabel{$(\tau_j :: A)$}
  \UnaryInfC{$\tau_i \subseteq^A \tau_j$} \DisplayProof
\item
  \AxiomC{$\tau_i :: A$} \AxiomC{$\tau_j \subseteq^A \tau_{j'}$}
  \LeftLabel{$(\conL)$}
  \BinaryInfC{$\tau_i \concat \tau_j \subseteq^A \tau_i \concat \tau_{j'}$}
  \DisplayProof
   \hskip 1.5em \AxiomC{$\tau_i \subseteq^A \tau_{i'}$}
  \AxiomC{$\tau_j :: A$} \LeftLabel{$(\conR)$}
  \BinaryInfC{$\tau_i \concat \tau_j \subseteq^A \tau_{i'} \concat \tau_j$}
  \DisplayProof
   \hskip 1.5em \AxiomC{$\tau_i \subseteq^A \tau_{k}$}
  \AxiomC{$\tau_j \subseteq^A \tau_{k}$} \LeftLabel{$(glb)$}
  \BinaryInfC{$\tau_i \concat \tau_j \subseteq^A \tau_k$} \DisplayProof
\item
  \AxiomC{$\tau_i \subseteq^A \tau_j$}
  \AxiomC{$\tau_{i'} \subseteq^A \tau_{j'}$} \LeftLabel{$(mon)$}
  \BinaryInfC{$\tau_i \concat \tau_{i'} \subseteq^A \tau_j \concat \tau_{j'}$}
  \DisplayProof
\item
  \AxiomC{$\tau_i :: A$} \AxiomC{$\tau_j :: A$} \LeftLabel{$(\tocap')$}
  \BinaryInfC{$\bigcap ((\tau_i \concat \tau_j) \leadsto (\tau_i \concat \tau_j)) \subseteq^{A \to B} \tau_i \leadsto \tau_i \cap \tau_j \leadsto \tau_j$}
  \DisplayProof
\end{enumerate}

\emph{Proof:}

\begin{enumerate}
\def\labelenumi{\roman{enumi})}
\item
  By induction on \(\tau\) and \(\tau_i\).
\item
  By induction on \(\tau_i \subseteq^A \tau_{i'}\).
\item
  \vskip 1em \AxiomC{$\tau_i \subseteq^A \tau_j$} \AxiomC{}
  \UnaryInfC{$\tau_j \subseteq \tau_j \concat \tau_{j'}$}
  \LeftLabel{$(\subseteq)$}
  \UnaryInfC{$\tau_j \subseteq^A \tau_j \concat \tau_{j'}$}
  \LeftLabel{$(trans)$}
  \BinaryInfC{$\tau_i \subseteq^A \tau_j \concat \tau_{j'}$}
  \AxiomC{$\tau_{i'} \subseteq^A \tau_{j'}$} \AxiomC{}
  \UnaryInfC{$\tau_{j'} \subseteq \tau_j \concat \tau_{j'}$}
  \LeftLabel{$(\subseteq)$}
  \UnaryInfC{$\tau_{j'} \subseteq^A \tau_j \concat \tau_{j'}$}
  \LeftLabel{$(trans)$}
  \BinaryInfC{$\tau_{i'} \subseteq^A \tau_j \concat \tau_{j'}$}
  \LeftLabel{$(glb)$}
  \BinaryInfC{$\tau_i \concat \tau_{i'} \subseteq^A \tau_j \concat \tau_{j'}$}
  \DisplayProof
\item
  Follows from \((\tocap)\), \((cons)\) and \((trans)\).
\end{enumerate}

\section{Intersection-type
assignment}\label{intersection-type-assignment}

Having annotated the \(\lambda\)-terms with simple types, the following
type assignment only permits the typing of simply-typed
\(\lambda\)-terms with an intersection type, which refines the simple
type of the \(\lambda\)-term:

\textbf{Definition} (Intersection-type assignment)

\begin{center}
  \AxiomC{$\exists (x, \tau_i, A) \in \Gamma.\ \bigcap \tau \subseteq^A \tau_i$}
  \LeftLabel{$(var)$}
  \UnaryInfC{$\Gamma \Vdash_s x_A : \tau$}
  \DisplayProof
  %------------------------------------
  \hskip 1.5em
  \AxiomC{$\Gamma \Vdash_s u_{A \to B} : \tau_i \leadsto \tau_j$}
  \AxiomC{$\Gamma \Vdash v_A : \tau_i$}
  \LeftLabel{$(app)$}
  \RightLabel{$(\bigcap \tau \subseteq^B \tau_j)$}
  \BinaryInfC{$\Gamma \Vdash_s uv_B : \tau$}
  \DisplayProof
  %------------------------------------
  \vskip 1.5em
  \AxiomC{$\forall x \not\in L.\ (x, \tau_i, A),\Gamma \Vdash m^x : \tau_j$}
  \LeftLabel{$(abs)$}
  \UnaryInfC{$\Gamma \Vdash_s \lambda_A.m : \tau_i \leadsto \tau_j$}
  \DisplayProof
  %------------------------------------
  \hskip 1.5em
  \AxiomC{$\exists \tau_x.\ \bigcap (\tau_x \leadsto \tau_x) \subseteq^{A \to A} \tau_i \land \tau_j \subseteq^A \tau_x$}
  \LeftLabel{$(Y)$}
  \UnaryInfC{$\Gamma \Vdash_s Y_{A} : \tau_i \leadsto \tau_j$}
  \DisplayProof
  %------------------------------------
  \vskip 1.5em
  \AxiomC{$\Gamma \Vdash_s m_{A \to B} : \tau_i \leadsto \tau_j$}
  \AxiomC{$\Gamma \Vdash_s m_{A \to B} : \tau_i \leadsto \tau_k$}
  \LeftLabel{$(\tocap)$}
  \RightLabel{$(\tau_{jk} \subseteq^B \tau_j \concat \tau_k)$}
  \BinaryInfC{$\Gamma \Vdash_s m_{A \to B} : \tau_i \leadsto \tau_{jk}$}
  \DisplayProof
  %------------------------------------
  \vskip 1.5em
  \AxiomC{}
  \LeftLabel{$(nil)$}
  \UnaryInfC{$\Gamma \Vdash m : \omega$}
  \DisplayProof
  %------------------------------------
  \hskip 1.5em
  \AxiomC{$\Gamma \Vdash_s m : \tau$}
  \AxiomC{$\Gamma \Vdash m : \tau_i$}
  \LeftLabel{$(cons)$}
  \BinaryInfC{$\Gamma \Vdash m : \tau , \tau_i$}
  \DisplayProof
  \vskip 1.5em
\end{center}

In the definition above, \(\Gamma\) is the typing context, consisting of
triples of the variable name and the corresponding intersection and
simple types. \(\Gamma\) is defined as a list of these triples in the
Agda implementation. It is assumed in the typing system, that \(\Gamma\)
is well-formed. Formally, this can be expressed in the following way:

\textbf{Definition} (Well-formed intersection-type context)

\begin{center}
  \AxiomC{}
  \LeftLabel{$(nil)$}
  \UnaryInfC{$\wf [\ ]$}
  \DisplayProof
  %------------------------------------
  \hskip 1.5em
  \AxiomC{$x \not\in \mathsf{dom}\ \Gamma$}
  \AxiomC{$\tau_i :: A$}
  \AxiomC{$\wf \Gamma$}
  \LeftLabel{$(cons)$}
  \TrinaryInfC{$\wf (x,\tau_i,A),\Gamma$}
  \DisplayProof
  \vskip 1.5em
\end{center}

\subsection{Subtyping}\label{subtyping}

In the typing system, the rules \((Y)\) and \((\tocap)\) are defined in
a slightly more complicated way than might be necessary. For example,
one might assume, the \((Y)\) rule could simply be:

\begin{center}
  \vskip 1em
  \AxiomC{}
  \LeftLabel{$(Y)$}
  \UnaryInfC{$\Gamma \Vdash_s Y_{A} : \bigcap (\tau_x \leadsto \tau_x) \leadsto \tau_x$}
  \DisplayProof
  \vskip 1.5em
\end{center}

The reason why the more complicated forms of both rules were introduced
was purely an engineering one, namely to make the proof of
sub-typing/weakening possible, as the sub-typing rule is required in
multiple further proofs:

\textbf{Lemma} (Sub-typing) The following rule(s) are admissible in
\(\Vdash_s\)/\(\Vdash\):

\begin{center}
  \AxiomC{$\Gamma \Vdash_s m_A : \tau$}
  \LeftLabel{$(\supseteq_s)$}
  \RightLabel{$(\Gamma' \subseteq_\Gamma \Gamma, \tau \supseteq^A_s \tau')$}
  \UnaryInfC{$\Gamma' \Vdash_s m_A : \tau'$}
  \DisplayProof
  \hskip 1.5em
  \AxiomC{$\Gamma \Vdash m_A : \tau_i$}
  \LeftLabel{$(\supseteq)$}
  \RightLabel{$(\Gamma' \subseteq_\Gamma \Gamma, \tau_i \supseteq^A_s \tau_j)$}
  \UnaryInfC{$\Gamma' \Vdash m_A : \tau_j$}
  \DisplayProof
\end{center}

\emph{Proof:} Ommited.

The relation \(\Gamma \subseteq_\Gamma \Gamma'\) is defined for any
well-formed contexts \(\Gamma, \Gamma'\), where for each triple
\((x ,\tau_i, A) \in \Gamma\), there is a corresponding triple
\((x ,\tau_j, A) \in \Gamma'\) s.t. \(\tau_i \subseteq^A \tau_j\).

\subsection{Inversion lemmas}\label{inversion-lemmas}

The shape of the derivation tree is not always unique for arbitrary
typed term \(\Gamma \Vdash_s m :\tau\). For example, given a typed term
\(\Gamma \Vdash_s \lambda_A.m :\tau_i \leadsto \tau_j\), either of the
following two derivation trees, could be valid:

\begin{center}\hskip 1.5em
  \AxiomC{$\vdots$}
  \UnaryInfC{$\forall x \not\in L.\ (x, \tau_i, A),\Gamma \Vdash m^x : \tau_j$}
  \LeftLabel{$(abs)$}
  \UnaryInfC{$\Gamma \Vdash_s \lambda_A.m : \tau_i \leadsto \tau_j$}
  \DisplayProof
  %------------------------------------
  \vskip 1.5em
  \AxiomC{$\vdots$}
  \UnaryInfC{$\Gamma \Vdash_s \lambda_A.m_B : \tau_i \leadsto \tau_p$}

  \AxiomC{$\vdots$}
  \UnaryInfC{$\Gamma \Vdash_s \lambda_A.m_B : \tau_i \leadsto \tau_q$}
  \LeftLabel{$(\tocap)$}
  \RightLabel{$(\tau_j \subseteq^B \tau_p \concat \tau_q)$}
  \BinaryInfC{$\Gamma \Vdash_s \lambda_A.m_B : \tau_i \leadsto \tau_j$}
  \DisplayProof
\end{center}

However, it is obvious that the second tree will always necessarily have
to have an application of \((abs)\) in all its branches. Because it will
be necessary to reason about the shape of the typing derivation trees,
it is useful to prove the following inversion lemmas:

\textbf{Lemma} (\(Y\)-inv, \(abs\)-inv)

\begin{enumerate}
\def\labelenumi{\roman{enumi})}
\tightlist
\item
  \(\Gamma \Vdash_s Y_{A} : \tau_i \leadsto \tau_j \implies \exists \tau_x.\ \bigcap (\tau_x \leadsto \tau_x) \subseteq^{A \to A} \tau_i \land \tau_j \subseteq^A \tau_x\)
\item
  \(\Gamma \Vdash_s \lambda_A.m : \tau_i \leadsto \tau_j \implies \exists L.\ \forall x \not\in L.\ (x, \tau_i, A),\Gamma \Vdash m^x : \tau_j\)
\end{enumerate}

\emph{Proof}:

\begin{enumerate}
\def\labelenumi{\roman{enumi})}
\item
  There are two cases to consider, one, where the last rule in the
  derivation tree of \(\Gamma \Vdash_s Y_{A} : \tau_i \leadsto \tau_j\)
  was \((Y)\). Otherwise, the last rule was \((\tocap)\):

  \((Y)\): Follows immediately.\\
  \((\tocap)\): We must have a derivation tree of the shape:

  \begin{center}
    \AxiomC{$\vdots$}
    \UnaryInfC{$\Gamma \Vdash_s Y_A : \tau_i \leadsto \tau_p$}

    \AxiomC{$\vdots$}
    \UnaryInfC{$\Gamma \Vdash_s Y_A : \tau_i \leadsto \tau_q$}
    \LeftLabel{$(\tocap)$}
    \RightLabel{$(\tau_j \subseteq^B \tau_p \concat \tau_q)$}
    \BinaryInfC{$\Gamma \Vdash_s Y_A : \tau_i \leadsto \tau_j$}
    \DisplayProof
  \end{center}

  Then by IH, we have:

  \begin{itemize}
  \item
    \(\exists \tau_{xp}.\ \bigcap (\tau_{xp} \leadsto \tau_{xp}) \subseteq^{A \to A} \tau_i \land \tau_p \subseteq^A \tau_{xp}\)
    and
  \item
    \(\exists \tau_{xq}.\ \bigcap (\tau_{xq} \leadsto \tau_{xq}) \subseteq^{A \to A} \tau_i \land \tau_q \subseteq^A \tau_{xq}\)
  \end{itemize}

  We then take \(\tau_x \equiv \tau_{xp} \concat \tau_{xq}\):
\end{enumerate}

\begin{center}
  \tiny
  \AxiomC{}
  \LeftLabel{$(\tocap')$}
  \UnaryInfC{$\bigcap (\tau_{x} \leadsto \tau_{x}) \subseteq^{A \to A} \tau_{xp}\leadsto \tau_{xp} \cap \tau_{xq} \leadsto \tau_{xq}$}

  \AxiomC{}
  \LeftLabel{$(IH)$}
  \UnaryInfC{$\tau_{xp} \leadsto \tau_{xp} \subseteq^{A \to A} \tau_i$}
  \AxiomC{}
  \LeftLabel{$(IH)$}
  \UnaryInfC{$\tau_{xq} \leadsto \tau_{xq} \subseteq^{A \to A} \tau_i$}
  \LeftLabel{$(mon)$}
  \BinaryInfC{$\tau_{xp}\leadsto \tau_{xp} \cap \tau_{xq} \leadsto \tau_{xq} \subseteq^{A \to A} \tau_i \concat \tau_i$}
  \LeftLabel{$(trans)$}
  \BinaryInfC{$\bigcap (\tau_{x} \leadsto \tau_{x}) \subseteq^{A \to A} \tau_i \concat \tau_i$}

  \AxiomC{}
  \UnaryInfC{$\tau_i \concat \tau_i \subseteq \tau_i$}
  \LeftLabel{$(\subseteq)$}
  \UnaryInfC{$\tau_i \concat \tau_i \subseteq^{A \to A} \tau_i$}
  \LeftLabel{$(trans)$}
  \BinaryInfC{$\bigcap (\tau_{x} \leadsto \tau_{x}) \subseteq^{A \to A} \tau_i$}
  \DisplayProof
\end{center}

\begin{center}
  \vskip 1.5em
  \AxiomC{}
  \UnaryInfC{$\tau_j \subseteq^A \tau_p \concat \tau_q$}

  \AxiomC{}
  \LeftLabel{$(IH)$}
  \UnaryInfC{$\tau_p \concat \subseteq^A \tau_{xp}$}
  \AxiomC{}
  \LeftLabel{$(IH)$}
  \UnaryInfC{$\tau_q \concat \subseteq^A \tau_{xq}$}
  \LeftLabel{$(mon)$}
  \BinaryInfC{$\tau_p \concat \tau_q \subseteq^A \tau_x$}
  \LeftLabel{$(trans)$}
  \BinaryInfC{$\tau_j \subseteq^A \tau_x$}
  \DisplayProof
  \vskip 1.5em
\end{center}

\begin{enumerate}
\def\labelenumi{\roman{enumi})}
\setcounter{enumi}{1}
\tightlist
\item
  Follows in a similar fashion.
\end{enumerate}

\section{Proofs of subject expansion and
reduction}\label{proofs-of-subject-expansion-and-reduction}

An interesting property of the intersection types, is the fact that they
admit both subject expansion and subject reduction, namely \(\Vdash\) is
closed under \(\beta\)-equality. Subject expansion and reduction are
proved in two separate lemmas:

\textbf{Theorem} (\(\Vdash\) closed under \(=_\beta\))

\begin{enumerate}
\def\labelenumi{\roman{enumi})}
\tightlist
\item
  \(\Gamma \Vdash_s m : \tau \implies m \Rightarrow_\beta m' \implies \Gamma \Vdash_s m' : \tau\)
\item
  \(\Gamma \Vdash m : \tau_i \implies m \Rightarrow_\beta m' \implies \Gamma \Vdash m' : \tau_i\)
\item
  \(\Gamma \Vdash_s m' : \tau \implies m \Rightarrow_\beta m' \implies \Gamma \Vdash_s m : \tau\)
\item
  \(\Gamma \Vdash m' : \tau_i \implies m \Rightarrow_\beta m' \implies \Gamma \Vdash m : \tau_i\)
\end{enumerate}

\emph{Proof:} By induction on \(\Rightarrow_\beta\). The proofs in both
directions follow by straightforward induction for all the rules except
for \((Y)\) and \((beta)\). Note that the \((Y)\) rule here is not the
typing rule, but rather the reduction rule
\(Y_A m \Rightarrow_\beta m(Y_A m)\).

\begin{enumerate}
\def\labelenumi{\roman{enumi})}
\tightlist
\item
  \((Y)\): By assumption, we have \(Y_A m \Rightarrow_\beta m(Y_A m)\)
  and \(\Gamma \Vdash_s Y_A m : \tau\). By case analysis of the last
  rule applied in the derivation tree of
  \(\Gamma \Vdash_s Y_A m : \tau\), we have two cases:

  \begin{itemize}
  \item
    \((app)\) We have:

    \begin{center}
      \vskip 1em
      \AxiomC{$\vdots$}
      \UnaryInfC{$\Gamma \Vdash_s Y_A : \tau_i \leadsto \tau_j$}
      \AxiomC{$\vdots$}
      \UnaryInfC{$\Gamma \Vdash m_{A \to A} : \tau_i$}
      \LeftLabel{$(app)$}
      \RightLabel{$(\bigcap \tau \subseteq^A \tau_j)$}
      \BinaryInfC{$\Gamma \Vdash_s Y_{A} m : \tau$}
      \DisplayProof
      \vskip 1em
    \end{center}

    Then, by (\(Y\)-inv) we have some \(\tau_x\) s.t
    \(\bigcap (\tau_x \leadsto \tau_x) \subseteq^{A \to A} \tau_i \land \tau_j \subseteq^A \tau_x\).
  \item
    \((\tocap)\) Then we have:

    \begin{center}
      \vskip 1em
      \AxiomC{$\vdots$}
      \UnaryInfC{$\Gamma \Vdash_s Y_{B \to C} m : \tau_i \leadsto \tau_j$}
      \AxiomC{$\vdots$}
      \UnaryInfC{$\Gamma \Vdash_s Y_{B \to C} m : \tau_i \leadsto \tau_k$}
      \LeftLabel{$(\tocap)$}
      \RightLabel{$(\tau_{jk} \subseteq^C \tau_j \concat \tau_k)$}
      \BinaryInfC{$\Gamma \Vdash_s Y_{B \to C} m : \tau_i \leadsto \tau_{jk}$}
      \DisplayProof
      \vskip 1em
    \end{center}

    Where \(A \equiv B \to C\).

    By IH, we get
    \(\Gamma \Vdash_s m (Y_{B \to C} m) : \tau_i \leadsto \tau_j\) and
    \(\Gamma \Vdash_s m (Y_{B \to C} m) : \tau_i \leadsto \tau_k\), thus
    from \((\tocap)\) it follows that
    \(\Gamma \Vdash_s m (Y_{B \to C} m) : \tau_i \leadsto \tau_{jk}\)
  \end{itemize}
\end{enumerate}

Test reference to lemma \ref{test}.

\footnotesize

\chapter*{References}\label{references}
\addcontentsline{toc}{chapter}{References}

\hypertarget{refs}{}
\hypertarget{ref-aydemir05}{}
Aydemir, Brian E., Aaron Bohannon, Matthew Fairbairn, J. Nathan Foster,
Benjamin C. Pierce, Peter Sewell, Dimitrios Vytiniotis, Geoffrey
Washburn, Stephanie Weirich, and Steve Zdancewic. 2005. ``Mechanized
Metatheory for the Masses: The Poplmark Challenge.'' In \emph{Theorem
Proving in Higher Order Logics: 18th International Conference, Tphols
2005, Oxford, Uk, August 22-25, 2005. Proceedings}, edited by Joe Hurd
and Tom Melham, 50--65. Berlin, Heidelberg: Springer Berlin Heidelberg.
doi:\href{https://doi.org/10.1007/11541868_4}{10.1007/11541868\_4}.

\hypertarget{ref-aydemir08}{}
Aydemir, Brian, Arthur Charguéraud, Benjamin C. Pierce, Randy Pollack,
and Stephanie Weirich. 2008. ``Engineering Formal Metatheory.'' In
\emph{Proceedings of the 35th Annual Acm Sigplan-Sigact Symposium on
Principles of Programming Languages}, 3--15. POPL '08. New York, NY,
USA: ACM.
doi:\href{https://doi.org/10.1145/1328438.1328443}{10.1145/1328438.1328443}.

\hypertarget{ref-bakel}{}
Bakel, Steffen van. 2003. ``Semantics with Intersection Types.''
\url{http://www.doc.ic.ac.uk/~svb/SemIntTypes/Notes.pdf}.

\hypertarget{ref-barendregt13}{}
Barendregt, Henk, Wil Dekkers, and Richard Statman. 2013. \emph{Lambda
Calculus with Types}. New York, NY, USA: Cambridge University Press.

\hypertarget{ref-berghofer06}{}
Berghofer, Stefan, and Christian Urban. 2006. ``A Head-to-Head
Comparison of de Bruijn Indices and Names.'' In \emph{IN Proc. Int.
Workshop on Logical Frameworks and Metalanguages: THEORY and Practice},
46--59.

\hypertarget{ref-clairambault13}{}
Clairambault, Pierre, and Andrzej S. Murawski. 2013. ``Böhm Trees as
Higher-Order Recursive Schemes.'' In \emph{IARCS Annual Conference on
Foundations of Software Technology and Theoretical Computer Science,
FSTTCS 2013, December 12-14, 2013, Guwahati, India}, 91--102.
doi:\href{https://doi.org/10.4230/LIPIcs.FSTTCS.2013.91}{10.4230/LIPIcs.FSTTCS.2013.91}.

\hypertarget{ref-harper93}{}
Harper, Robert, Furio Honsell, and Gordon Plotkin. 1993. ``A Framework
for Defining Logics.'' \emph{J. ACM} 40 (1). New York, NY, USA: ACM:
143--84.
doi:\href{https://doi.org/10.1145/138027.138060}{10.1145/138027.138060}.

\hypertarget{ref-kobayashi13}{}
Kobayashi, Naoki. 2013. ``Model Checking Higher-Order Programs.''
\emph{J. ACM} 60 (3). New York, NY, USA: ACM: 20:1--20:62.
doi:\href{https://doi.org/10.1145/2487241.2487246}{10.1145/2487241.2487246}.

\hypertarget{ref-shing-cheng}{}
Mu, Shin-Cheng. 2011. ``Proving the Church-Rosser Theorem Using a
Locally Nameless Representation.'' Blog.
\url{http://www.iis.sinica.edu.tw/~scm/2011/proving-the-church-rosser-theorem}.

\hypertarget{ref-ong06}{}
Ong, C.-H. L. 2006. ``On Model-Checking Trees Generated by Higher-Order
Recursion Schemes.'' In \emph{Proceedings of the 21st Annual Ieee
Symposium on Logic in Computer Science}, 81--90. LICS '06. Washington,
DC, USA: IEEE Computer Society.
doi:\href{https://doi.org/10.1109/LICS.2006.38}{10.1109/LICS.2006.38}.

\hypertarget{ref-pfenning88}{}
Pfenning, F., and C. Elliott. 1988. ``Higher-Order Abstract Syntax.'' In
\emph{Proceedings of the Acm Sigplan 1988 Conference on Programming
Language Design and Implementation}, 199--208. PLDI '88. New York, NY,
USA: ACM.
doi:\href{https://doi.org/10.1145/53990.54010}{10.1145/53990.54010}.

\hypertarget{ref-pfenning99}{}
Pfenning, Frank, and Carsten Schürmann. 1999. ``Automated Deduction ---
Cade-16: 16th International Conference on Automated Deduction Trento,
Italy, July 7--10, 1999 Proceedings.'' In, 202--6. Berlin, Heidelberg:
Springer Berlin Heidelberg.
doi:\href{https://doi.org/10.1007/3-540-48660-7_14}{10.1007/3-540-48660-7\_14}.

\hypertarget{ref-pollack95}{}
Pollack, Robert. 1995. ``Polishing up the Tait-Martin-Löf Proof of the
Church-Rosser Theorem.''

\hypertarget{ref-ramsay14}{}
Ramsay, Steven J., Robin P. Neatherway, and C.-H. Luke Ong. 2014. ``A
Type-Directed Abstraction Refinement Approach to Higher-Order Model
Checking.'' \emph{SIGPLAN Not.} 49 (1). New York, NY, USA: ACM: 61--72.
doi:\href{https://doi.org/10.1145/2578855.2535873}{10.1145/2578855.2535873}.

\hypertarget{ref-takahashi95}{}
Takahashi, M. 1995. ``Parallel Reductions in \(\lambda\)-Calculus.''
\emph{Information and Computation} 118 (1): 120--27.
\url{http://www.sciencedirect.com/science/article/pii/S0890540185710577}.

\hypertarget{ref-tsukada14}{}
Tsukada, Takeshi, and C.-H. Luke Ong. 2014. ``Compositional Higher-Order
Model Checking via \$\$-Regular Games over Böhm Trees.'' In
\emph{Proceedings of the Joint Meeting of the Twenty-Third Eacsl Annual
Conference on Computer Science Logic (Csl) and the Twenty-Ninth Annual
Acm/Ieee Symposium on Logic in Computer Science (Lics)}, 78:1--78:10.
CSL-Lics '14. New York, NY, USA: ACM.
doi:\href{https://doi.org/10.1145/2603088.2603133}{10.1145/2603088.2603133}.

% - Back matter ----------------------------------------------------------------




\end{document}
