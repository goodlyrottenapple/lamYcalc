\documentclass[a4paperpaper,]{article}
\usepackage{lmodern}
\usepackage{amssymb,amsmath}
\usepackage{ifxetex,ifluatex}
\usepackage{fixltx2e} % provides \textsubscript
\ifnum 0\ifxetex 1\fi\ifluatex 1\fi=0 % if pdftex
  \usepackage[T1]{fontenc}
  \usepackage[utf8]{inputenc}
\else % if luatex or xelatex
  \ifxetex
    \usepackage{mathspec}
  \else
    \usepackage{fontspec}
  \fi
  \defaultfontfeatures{Ligatures=TeX,Scale=MatchLowercase}
    \setmainfont[]{Fira Sans Light}
    \setsansfont[]{Fira Sans}
    \setmonofont[Mapping=tex-ansi,Scale=0.8]{FreeMono}
    \setmathfont(Digits,Latin,Greek)[]{Fira Sans Light}
\fi
% use upquote if available, for straight quotes in verbatim environments
\IfFileExists{upquote.sty}{\usepackage{upquote}}{}
% use microtype if available
\IfFileExists{microtype.sty}{%
\usepackage{microtype}
\UseMicrotypeSet[protrusion]{basicmath} % disable protrusion for tt fonts
}{}
\usepackage[margin=1in]{geometry}
\usepackage{hyperref}
\PassOptionsToPackage{usenames,dvipsnames}{color} % color is loaded by hyperref
\hypersetup{unicode=true,
            pdftitle={A formalization of the \textbackslash{}lamy calculus},
            pdfauthor={Samuel Balco},
            colorlinks=true,
            linkcolor=cyan,
            citecolor=Blue,
            urlcolor=cyan,
            breaklinks=true}
\urlstyle{same}  % don't use monospace font for urls
% Make links footnotes instead of hotlinks:
\renewcommand{\href}[2]{#2\footnote{\url{#1}}}
\IfFileExists{parskip.sty}{%
\usepackage{parskip}
}{% else
\setlength{\parindent}{0pt}
\setlength{\parskip}{6pt plus 2pt minus 1pt}
}
\setlength{\emergencystretch}{3em}  % prevent overfull lines
\providecommand{\tightlist}{%
  \setlength{\itemsep}{0pt}\setlength{\parskip}{0pt}}
\setcounter{secnumdepth}{0}
% Redefines (sub)paragraphs to behave more like sections
\ifx\paragraph\undefined\else
\let\oldparagraph\paragraph
\renewcommand{\paragraph}[1]{\oldparagraph{#1}\mbox{}}
\fi
\ifx\subparagraph\undefined\else
\let\oldsubparagraph\subparagraph
\renewcommand{\subparagraph}[1]{\oldsubparagraph{#1}\mbox{}}
\fi
\usepackage{bussproofs}
\usepackage{amsthm}
\usepackage{minted}
\let\OldTexttt\texttt
\renewcommand{\texttt}[1]{\small\OldTexttt{#1}}
\newcommand{\lamy}{\lambda\text{-}Y}
\newcommand{\concat}{\ensuremath{+\!\!\!\!+\,}}
\newcommand{\wf}{\textsf{Wf-ICtxt}\ }
\newcommand{\tocap}{\leadsto\kern-.5ex\cap}
\newcommand{\conR}{\concat_{\kern-1ex R}}
\newcommand{\conL}{\concat_{\kern-1ex L}}
\newcommand{\poplm}{\textsc{PoplMark}}
\renewcommand{\max}{\textsf{max}\ }

\title{A formalization of the \(\lamy\) calculus}
\author{Samuel Balco}
\date{}

\begin{document}
\maketitle
\begin{abstract}
This is the abstract. For this and the other front-matter options you
can either include the text directly on the metadata file or you can use
in order to include your text.
\end{abstract}

{
\hypersetup{linkcolor=black}
\setcounter{tocdepth}{2}
\tableofcontents
}
\section{Introduction}\label{introduction}

Formal verification of software is a field of active research in
computer science. One of the main approaches to verification is model
checking, wherein a system specification is checked against certain
correctness properties, by finding a model of the system, encoding the
desired correctness property as a logical formula and then exhaustively
checking whether the given formula is satisfiable in the model of the
system. Big advances in model checking of 1\textsuperscript{st} order
(imperative) programs have been made, with techniques like abstraction
refinement and SAT/SMT-solver use, allowing scalability.\\
Higher order (functional) program verification, on the other hand, has
been much less explored. Current approaches to formal verification of
such programs usually involve the use of (automatic) theorem provers,
which usually require a lot of user interaction and as a result have not
managed to scale as well as model checking in the 1\textsuperscript{st}
order setting. In recent years, advances in higher order model checking
(HOMC) have been made by Ong / ? (find paper??). Whilst a lot of theory
has been developed for HOMC, there has been little done in
implementing/mechanizing these results in a fully formal setting of a
theorem prover.\\
The aim of this project is to make a start of such a mechanization, by
formalizing the \(\lamy\) calculus with the intersection-type system
described by ? and formally proving important properties of the
system.\\
The first part of this work focuses on the mechanization aspect of the
simply typed \(\lamy\) calculus in a theorem prover, in a fashion
similar to the \(\poplm\) challenge, by exploring different
formalizations of the calculus and the use of different theorem provers.
The project focuses on the engineering choices and formalization
overheads, which result from translating the informal systems into a
fully-formal setting of a theorem prover.

\subsection{Binders}\label{binders}

When describing the (untyped) \(\lambda\)-calculus on paper, the terms
of the \(\lambda\)-calculus are usually inductively defined in the
following way:

\[t::= x\ |\ tt\ |\ \lambda x.t \text{ where }x \in Var\]

This definition of terms yields an induction/recursion principle, which
can be used to define functions over the \(\lambda\)-terms by structural
recursion and prove properties about the \(\lambda\)-terms using
structural induction (recursion and induction being two sides of the
same coin).\\
However, whilst the definition above describes valid terms of the
\(\lambda\)-calculus, there are implicit assumptions one makes about the
terms, namely, the \(x\) in the \(\lambda x.t\) case appears bound in
\(t\). This means that while \(x\) and \(y\) might be distinct terms of
the \(\lambda\)-calculus (i.e. \(x \neq y\)), \(\lambda x.x\) and
\(\lambda y.y\) represent the same term, as \(x\) and \(y\) are bound by
the \(\lambda\). Without the notion of \(\alpha\)-equivalence of terms,
one cannot prove any properties of terms involving bound variables, such
as saying that \(\lambda x.x \equiv \lambda y.y\).

In an informal setting, reasoning with \(\alpha\)-equivalence of terms
is often very implicit, however in a formal setting of theorem provers,
having an inductive definition of ``raw'' \(lambda\)-terms, which are
not \(alpha\)-equivalent, yet reasoning about \(\alpha\)-equivalent
\(\lambda\)-terms poses certain challenges.\\
One of the main problems is the fact that the inductive/recursive
definition does not easily lift to \(alpha\)-equivalent terms. Take a
trivial example of a function on raw terms, which checks whether a
variable appears bound in a given \(\lambda\)-term. Clearly, such
function is well formed for ``raw'' terms, but does not work (or even
make sense) for \(\alpha\)-equivalent terms.\\
Conversely, there are informal definitions over \(\alpha\)-equivalent
terms, which are not straight-forward to define over raw terms. Take the
usual definition of substitution, defined over \(\alpha\)-equivalent
terms, which actually relies on this fact in the following case:

\[(\lambda y'. s')[t/x] \equiv \lambda y'.(s'[t/x]) \text{ assuming } y' \not\equiv x\text{ and }y' \not\in FV(t)\]

Here in the \(\lambda\) case, it is assumed that a given lambda term
\(\lambda y. s\) can always be swapped out for an alpha equivalent term
\(\lambda y'. s'\), such that \(y'\) satisfies the side condition. The
assumption that a bound variable can be swapped out for a ``fresh'' one
to avoid name clashes is often referred to as the Barendregt Variable
Convention.

The direct approach of defining ``raw'' terms and an additional notion
of \(\alpha\)-equivalence introduces a lot of overhead when defining
functions, as one either has to use the recursive principles for ``raw''
terms and then show that the function lifts to the \(\alpha\)-equivalent
terms or define functions on \(alpha\)-equivalence classes and prove
that it is well-founded, without being able to rely on the structurally
inductive principles that one gets ``for free'' with the ``raw''
terms.\\
Because of this, the usual informal representation of the
\(\lambda\)-calculus is rarely used in a fully formal setting.

To mitigate the overheads of a fully formal definition of the
\(\lambda\)-calculus, we want to have an encoding of the
\(\lambda\)-terms, which includes the notion of \(\alpha\)-equivalence
whilst being inductively defined, giving us the inductive/recursive
principles for \(alpha\)-equivalent terms directly. This can be achieved
in several different ways. In general, there are two main approaches
taken in a rigorous formalization of the terms of the lambda calculus,
namely the concrete approaches and the higher-order approaches, both
described in some detail below.

\subsubsection{Concrete approaches}\label{concrete-approaches}

The concrete or first-order approaches usually encode variables using
names (like strings or natural numbers). Encoding of terms and
capture-avoiding substitution must be encoded explicitly. A survey by
Aydemir et al. (\protect\hyperlink{ref-aydemir08}{2008}) details three
main groups of concrete approaches, found in formalizations of the
\(\lambda\)-calculus in the literature:

\paragraph{Named}\label{named}

This approach generally defines terms in much the same way as the
informal inductive definition given above. Using a functional language,
such as Haskell or ML, such a definition might look like this:

\begin{minted}[]{isabelle}
datatype trm =
  Var name
| App trm trm
| Lam name trm
\end{minted}

As was mentioned before, defining ``raw'' terms and the notion of
\(\alpha\)-equivalence of ``raw'' terms separately carries a lot of
overhead in a theorem prover and is therefore not favored.

To obtain an inductive definition of \(\lambda\)-terms with a built in
notion of \(\alpha\)-equivalence, one can instead use nominal sets
(described in the section on nominal sets/Isabelle?). The nominal
package in Isabelle provides tools to automatically define terms with
binders, which generate inductive definitions of \(\alpha\)-equivalent
terms. Using nominal sets in Isabelle results in a definition of terms
which looks very similar to the informal presentation of the lambda
calculus:

\begin{minted}[]{isabelle}
nominal_datatype trm =
  Var name
| App trm trm
| Lam x::name l::trm  binds x in l
\end{minted}

Most importantly, this definition allows one to define functions over
\(\alpha\)-equivalent terms using structural induction. The nominal
package also provides freshness lemmas and a strengthened induction
principle with name freshness for terms involving binders.

\paragraph{Nameless/de Bruijn}\label{namelessde-bruijn}

Using a named representation of the lambda calculus in a fully formal
setting can be inconvenient when dealing with bound variables. For
example, substitution, as described in the introduction, with its
side-condition of freshness of \(y\) in \(x\) and \(t\) is not
structurally recursive on ``raw'' terms, but rather requires
well-founded recursion over \(\alpha\)-equivalence classes of terms. To
avoid this problem in the definition of substitution, the terms of the
lambda calculus can be encoded using de Bruijn indices:

\begin{minted}[]{isabelle}
datatype trm =
  Var nat
| App trm trm
| Lam trm
\end{minted}

This representation of terms uses indices instead of named variables.
The indices are natural numbers, which encode an occurrence of a
variable in a \(\lambda\)-term. For bound variables, the index indicates
which \(\lambda\) it refers to, by encoding the number of
\(\lambda\)-binders that are in the scope between the index and the
\(\lambda\)-binder the variable corresponds to. For example, the term
\(\lambda x.\lambda y. yx\) will be represented as
\(\lambda\ \lambda\ 0\ 1\). Here, 0 stands for \(y\), as there are no
binders in scope between itself and the \(\lambda\) it corresponds to,
and \(1\) corresponds to \(x\), as there is one \(\lambda\)-binder in
scope. To encode free variables, one simply choses an index greater than
the number of \(\lambda\)'s currently in scope, for example,
\(\lambda\ 4\).

To see that this representation of \(\lambda\)-terms is isomorphic to
the usual named definition, we can define two function \(f\) and \(g\),
which translate the named representation to de Bruijn notation and vice
versa. More precisely, since we are dealing with \(\alpha\)-equivalence
classes, its is an isomorphism between these that we can formalize.

To make things easier, we consider a representation of named terms,
where we map named variables, \(x, y, z,...\) to indexed variables
\(x_1,x_2,x_3,...\). Then, the mapping from named terms to de Bruijn
term is given by \(f\), which we define in terms of an auxiliary
function \(e\):

\begin{align*} 
e_k^m(x_n) &= \begin{cases}
k-m(x_n)-1 & x_n \in \text{dom }m\\
k+n & otherwise
\end{cases}\\
e_k^m(uv) &= e_k^m(u)\ e_k^m(v)\\
e_k^m(\lambda x_n.u) &= \lambda\ e_{k+1}^{m \oplus (x_n,k)}(u)
\end{align*}

Then \(f(t) \equiv e_0^\emptyset(t)\)

The function \(e\) takes two additional parameters, \(k\) and \(m\).
\(k\) keeps track of the scope from the root of the term and \(m\) is a
map from bound variables to the levels they were bound at. In the
variable case, if \(x_n\) appears in \(m\), it is a bound variable, and
it's index can be calculated by taking the difference between the
current index and the index \(m(x_k)\), at which the variable was bound.
If \(x_n\) is not in \(m\), then the variable is encoded by adding the
current level \(k\) to \(n\).\\
In the abstraction case, \(x_n\) is added to \(m\) with the current
level \(k\), possibly overshadowing a previous binding of the same
variable at a different level (like in
\(\lambda x_1. (\lambda x_1. x_1)\)) and \(k\) is incremented, going
into the body of the abstraction.

The function \(g\), taking de Bruijn terms to named terms is a little
more tricky. We need to replace indices encoding free variables (those
that have a value greater than or equal to \(k\), where \(k\) is the
number of binders in scope) with named variables, such that for every
index \(n\), we substitute \(x_m\), where \(m = n-k\), without capturing
these free variables.

We need two auxiliary functions to define \(g\):

\begin{align*} 
h_k^b(n) &= \begin{cases}
x_{n-k} & n \geq k\\
x_{k+b-n-1} & otherwise
\end{cases}\\
h_k^b(uv) &= h_k^b(u)\ h_k^b(v)\\
h_k^b(\lambda u) &= \lambda x_{k+b}.\ h_{k+1}^b(u)
\end{align*}

\begin{align*} 
\Diamond_k(n) &= \begin{cases}
n-k & n \geq k\\
0 & otherwise
\end{cases}\\
\Diamond_k(uv) &= \max (\Diamond_k(u),\ \Diamond_k(v))\\
\Diamond_k(\lambda u) &= \Diamond_{k+1}(u)
\end{align*}

The function \(g\) is then defined as
\(g(t) \equiv h_0^{\Diamond_0(t)+1}(t)\). As mentioned above, the
complicated definition has to do with avoiding free variable capture. A
term like \(\lambda (\lambda\ 2)\) intuitively represents a named lambda
term with two bound variables and a free variable \(x_0\) according to
the definition above. If we started giving the bound variables names in
a naive way, starting from \(x_0\), we would end up with a term
\(\lambda x_0.(\lambda x_1.x_0)\), which is obviously not the term we
had in mind, as \(x_0\) is no longer a free variable. To ensure we start
naming the bound variables in such a way as to avoid this situation, we
use \(\Diamond\) to compute the maximal value of any free variable in
the given term, and then start naming bound variables with an index one
higher than the value returned by \(\Diamond\).

As one quickly notices, a term like \(\lambda x.x\) and \(\lambda y.y\)
have a single unique representation as a \(de Bruijn term\)
\(\lambda\ 0\). Indeed, since there are no named variables in a de
Bruijn term, there is only one way to represent any \(\lambda\)-term,
and the notion of \(\alpha\)-equivalence is no longer relevant. We thus
get around our problem of having an inductive principle and
\(\alpha\)-equivalent terms, by having a representation of
\(\lambda\)-terms where every \(\alpha\)-equivalence class of
\(\lambda\)-terms has a single representative term in the de Bruijn
notation.

In their comparison between named vs.~nameless/de Bruijn representations
of lambda terms, Berghofer and Urban
(\protect\hyperlink{ref-berghofer06}{2006}) give details about the
definition of substitution, which no longer needs the variable
convention and can therefore be defined using primitive structural
recursion.\\
The main disadvantage of using de Bruijn indices is the relative
unreadability of both the terms and the formulation of properties about
these terms. For example, the substitution lemma, which in the named
setting would be stated as:

\[\text{If }x \neq y\text{ and }x \not\in FV(L)\text{, then }
M[N/x][L/y] \equiv M[L/y][N[L/y]/x].\]

becomes the following statement in the nameless formalization:

\[\text{For all indices }i, j\text{ with }i \leq j\text{, }M[N/i][L/j] = M[L/j + 1][N[L/j - i]/i]\]

Clearly, the first version of this lemma is much more intuitive.

\paragraph{Locally Nameless}\label{locally-nameless}

The locally nameless approach to binders is a mix of the two previous
approaches. Whilst a named representation uses variables for both free
and bound variables and the nameless encoding uses de Bruijn indices in
both cases as well, a locally nameless encoding distinguishes between
the two types of variables.\\
Free variables are represented by names, much like in the named version,
and bound variables are encoded using de Bruijn indices. By using de
Bruijn indices for bound variables, we again obtain an inductive
definition of terms which are already \(alpha\)-equivalent.

While closed terms, like \(\lambda x.x\) and \(\lambda y.y\) are
represented as de Bruijn terms, the term \(\lambda x.xz\) and
\(\lambda x.xz\) are encoded as \(\lambda\ 0z\). The following
definition captures the syntax of the locally nameless terms:

\begin{minted}[]{isabelle}
datatype ptrm =
  Fvar name
  BVar nat
| App trm trm
| Lam trm
\end{minted}

Note however, that this definition doesn't quite fit the notion of
\(\lambda\)-terms, since a \texttt{pterm} like \texttt{(BVar 1)} does
not represent a \(\lambda\)-term, since bound variables can only appear
in the context of a lambda, such as in \texttt{(Lam (BVar 1))}.

The advantage of using a locally nameless definition of
\(\lambda\)-terms is a better readability of such terms, compared to
equivalent de Bruijn terms. Another advantage is the fact that
definitions of functions and reasoning about properties of these terms
is much closer to the informal setting.

\subsubsection{Higher-Order approaches}\label{higher-order-approaches}

Unlike concrete approaches to formalizing the lambda calculus, where the
notion of binding and substitution is defined explicitly in the host
language, higher-order formalizations use the function space of the
implementation language, which handles binding. HOAS, or higher-order
abstract syntax (F. Pfenning and Elliott
\protect\hyperlink{ref-pfenning88}{1988}, Harper, Honsell, and Plotkin
(\protect\hyperlink{ref-harper93}{1993})), is a framework for defining
logics based on the simply typed lambda calculus. A form of HOAS,
introduced by Harper, Honsell, and Plotkin
(\protect\hyperlink{ref-harper93}{1993}), called the Logical Framework
(LF) has been implemented as Twelf by Frank Pfenning and Schürmann
(\protect\hyperlink{ref-pfenning99}{1999}), which has been previously
used to encode the \(\lambda\)-calculus.\\
Using HOAS for encoding the \(\lambda\)-calculus comes down to encoding
binders using the meta-language binders. This way, the definitions of
capture avoiding substitution or notion of \(\alpha\)-equivalence are
offloaded onto the meta-language. As an example, take the following
definition of terms of the \(\lambda\)-calculus in Haskell:

\begin{minted}[]{haskell}
data Term where
  Var :: Int -> Term
  App :: Term -> Term -> Term
  Lam :: (Term -> Term) -> Term
\end{minted}

This definition avoids the need for explicitly defining substitution,
because it encodes a lambda term as a Haskell function
\texttt{(Term -> Term)}, relying on Haskell's internal substitution and
notion of \(\alpha\)-equivalence. As with the de Bruijn and locally
nameless representations, this encoding gives us inductively defined
terms with a built in notion of \(\alpha\)-equivalence.\\
However, using HOAS only works if the notion of \(\alpha\)-equivalence
and substitution of the meta-language coincide with these notions in the
object-language.

\section{Methodology}\label{methodology}

\subsection{\texorpdfstring{\(\lambda\)-Y calculus -
Definitions}{\textbackslash{}lambda-Y calculus - Definitions}}\label{lambda-y-calculus---definitions}

\textbf{Syntax (nominal)}

Types:
\[\sigma ::= a\ |\ \sigma \to \sigma \text{ where }a \in \mathcal{TV}\]

Terms:
\[M::= x\ |\ MM\ |\ \lambda x.M\ |\ Y_\sigma \text{ where }x \in Var\]

\textbf{Well typed terms (nominal)}

\begin{center}
    \AxiomC{}
    \LeftLabel{$(var)$}
    \RightLabel{$(x : \sigma \in \Gamma)$}
    \UnaryInfC{$\Gamma \vdash x : \sigma$}
    \DisplayProof
    \hskip 1.5em
    \AxiomC{}
    \LeftLabel{$(Y)$}
    \UnaryInfC{$\Gamma \vdash Y_\sigma : (\sigma \to \sigma) \to \sigma$}
    \DisplayProof
    \vskip 1.5em
    \AxiomC{$\Gamma \cup \{x:\sigma\} \vdash M : \tau$}
    \LeftLabel{$(abs)$}
    \RightLabel{$(x\ \sharp\ \Gamma/ x \not\in Subjects(\Gamma))$}
    \UnaryInfC{$\Gamma \vdash \lambda x. M : \sigma \to \tau$}
    \DisplayProof
    \hskip 1.5em
    \AxiomC{$\Gamma \vdash M : \sigma \to \tau$}
    \AxiomC{$\Gamma \vdash N : \sigma$}
    \LeftLabel{$(app)$}
    \BinaryInfC{$\Gamma \vdash MN : \tau$}
    \DisplayProof
\end{center}

\textbf{\(\beta Y\)-Reduction(nominal, typed)}

\begin{center}
    \AxiomC{$\Gamma \vdash M \Rightarrow M' : \sigma \to \tau$}
    \AxiomC{$\Gamma \vdash N : \sigma$}
    \LeftLabel{$(red_L)$}
    \BinaryInfC{$\Gamma \vdash MN \Rightarrow M'N : \tau$}
    \DisplayProof
    \hskip 1.5em
    \AxiomC{$\Gamma \vdash M : \sigma \to \tau$}
    \AxiomC{$\Gamma \vdash N \Rightarrow N' : \sigma$}
    \LeftLabel{$(red_R)$}
    \BinaryInfC{$\Gamma \vdash MN \Rightarrow M'N : \tau$}
    \DisplayProof
    \vskip 1.5em
    \AxiomC{$\Gamma \cup \{x:\sigma\} \vdash M \Rightarrow M' : \tau$}
    \LeftLabel{$(abs)$}
    \RightLabel{$(x\ \sharp\ \Gamma)$}
    \UnaryInfC{$\Gamma \vdash \lambda x. M \Rightarrow \lambda x. M' : \sigma \to \tau$}
    \DisplayProof
    \hskip 1.5em
    \AxiomC{$\Gamma \cup \{x:\sigma\} \vdash M : \tau$}
    \AxiomC{$\Gamma \vdash N : \sigma$}
    \LeftLabel{$(\beta)$}
    \RightLabel{$(x\ \sharp\ \Gamma, N)$}
    \BinaryInfC{$\Gamma \vdash (\lambda x. M)N \Rightarrow M[N/x] : \tau$}
    \DisplayProof
    \vskip 1.5em
    \AxiomC{$\Gamma \vdash M : \sigma \to \sigma$}
    \LeftLabel{$(Y)$}
    \UnaryInfC{$\Gamma \vdash Y_\sigma M \Rightarrow M (Y_\sigma M) : \sigma$}
    \DisplayProof
\end{center}

\textbf{\(\beta Y\)-Reduction(nominal, untyped)}

\begin{center}
    \AxiomC{$M \Rightarrow M'$}
    \LeftLabel{$(red_L)$}
    \UnaryInfC{$MN \Rightarrow M'N$}
    \DisplayProof
    \hskip 1.5em
    \AxiomC{$N \Rightarrow N'$}
    \LeftLabel{$(red_R)$}
    \UnaryInfC{$MN \Rightarrow M'N$}
    \DisplayProof
    \vskip 1.5em
    \AxiomC{$M \Rightarrow M'$}
    \LeftLabel{$(abs)$}
    \UnaryInfC{$\lambda x. M \Rightarrow \lambda x. M'$}
    \DisplayProof
    \hskip 1.5em
    \AxiomC{}
    \LeftLabel{$(\beta)$}
    \RightLabel{$(x\ \sharp\ N)$}
    \UnaryInfC{$(\lambda x. M)N \Rightarrow M[N/x]$}
    \DisplayProof
    \hskip 1.5em
    \AxiomC{}
    \LeftLabel{$(Y)$}
    \UnaryInfC{$M \Rightarrow M (Y_\sigma M)$}
    \DisplayProof
\end{center}

\begin{center}\rule{0.5\linewidth}{\linethickness}\end{center}

\textbf{Syntax (locally nameless)}

Types:
\[\sigma ::= a\ |\ \sigma \to \sigma \text{ where }a \in \mathcal{TV}\]

Pre-terms:
\[M::= x\ |\ n\ |\ MM\ |\ \lambda M\ |\ Y_\sigma \text{ where }x \in Var \text{ and } n \in Nat\]

\textbf{Open (locally nameless)}

\(M^N \equiv \{0 \to N\}M\\\)

\begin{math}
\{k \to U\}(x) = x\\
\{k \to U\}(n) = \text{if }k = n \text{ then } U \text{ else } n\\
\{k \to U\}(MN) = (\{k \to U\}M)(\{k \to U\}N)\\
\{k \to U\}(\lambda M) = \lambda (\{k+1 \to U\}M)\\
\{k \to U\}(Y \sigma) = Y \sigma
\end{math}

\textbf{Closed terms (locally nameless, cofinite)}

\begin{center}
    \AxiomC{}
    \LeftLabel{$(fvar)$}
    \UnaryInfC{term$(x)$}
    \DisplayProof
    \hskip 1.5em
    \AxiomC{}
    \LeftLabel{$(Y)$}
    \UnaryInfC{term$(Y_\sigma)$}
    \DisplayProof
    \vskip 1.5em
    \AxiomC{$\forall x \not\in L.\ \text{term}(M^x) $}
    \LeftLabel{$(lam)$}
    \RightLabel{(finite $L$)}
    \UnaryInfC{term$(\lambda M)$}
    \DisplayProof
    \hskip 1.5em
    \AxiomC{term$(M)$}
    \AxiomC{term$(M)$}
    \LeftLabel{$(app)$}
    \BinaryInfC{term$(MN)$}
    \DisplayProof
\end{center}

\textbf{\(\beta Y\)-Reduction(locally nameless, cofinite, untyped)}

\begin{center}
    \AxiomC{$M \Rightarrow M'$}
    \AxiomC{term$(N)$}
    \LeftLabel{$(red_L)$}
    \BinaryInfC{$MN \Rightarrow M'N$}
    \DisplayProof
    \hskip 1.5em
    \AxiomC{term$(M)$}
    \AxiomC{$N \Rightarrow N'$}
    \LeftLabel{$(red_R)$}
    \BinaryInfC{$MN \Rightarrow M'N$}
    \DisplayProof
    \vskip 1.5em
    \AxiomC{$\forall x \not\in L.\ M^x \Rightarrow M'^x$}
    \LeftLabel{$(abs)$}
    \RightLabel{(finite $L$)}
    \UnaryInfC{$\lambda M \Rightarrow \lambda M'$}
    \DisplayProof
    \hskip 1.5em
    \AxiomC{term$(\lambda M)$}
    \AxiomC{term$(N)$}
    \LeftLabel{$(\beta)$}
    \BinaryInfC{$(\lambda M)N \Rightarrow M^N$}
    \DisplayProof
    \hskip 1.5em
    \AxiomC{}
    \LeftLabel{$(Y)$}
    \UnaryInfC{$M \Rightarrow M (Y_\sigma M)$}
    \DisplayProof
\end{center}

\hypertarget{refs}{}
\hypertarget{ref-aydemir08}{}
Aydemir, Brian, Arthur Charguéraud, Benjamin C. Pierce, Randy Pollack,
and Stephanie Weirich. 2008. ``Engineering Formal Metatheory.'' In
\emph{Proceedings of the 35th Annual Acm Sigplan-Sigact Symposium on
Principles of Programming Languages}, 3--15. POPL '08. New York, NY,
USA: ACM.
doi:\href{https://doi.org/10.1145/1328438.1328443}{10.1145/1328438.1328443}.

\hypertarget{ref-berghofer06}{}
Berghofer, Stefan, and Christian Urban. 2006. ``A Head-to-Head
Comparison of de Bruijn Indices and Names.'' In \emph{IN Proc. Int.
Workshop on Logical Frameworks and Metalanguages: THEORY and Practice},
46--59.

\hypertarget{ref-harper93}{}
Harper, Robert, Furio Honsell, and Gordon Plotkin. 1993. ``A Framework
for Defining Logics.'' \emph{J. ACM} 40 (1). New York, NY, USA: ACM:
143--84.
doi:\href{https://doi.org/10.1145/138027.138060}{10.1145/138027.138060}.

\hypertarget{ref-pfenning88}{}
Pfenning, F., and C. Elliott. 1988. ``Higher-Order Abstract Syntax.'' In
\emph{Proceedings of the Acm Sigplan 1988 Conference on Programming
Language Design and Implementation}, 199--208. PLDI '88. New York, NY,
USA: ACM.
doi:\href{https://doi.org/10.1145/53990.54010}{10.1145/53990.54010}.

\hypertarget{ref-pfenning99}{}
Pfenning, Frank, and Carsten Schürmann. 1999. ``Automated Deduction ---
Cade-16: 16th International Conference on Automated Deduction Trento,
Italy, July 7--10, 1999 Proceedings.'' In, 202--6. Berlin, Heidelberg:
Springer Berlin Heidelberg.
doi:\href{https://doi.org/10.1007/3-540-48660-7_14}{10.1007/3-540-48660-7\_14}.

\end{document}
